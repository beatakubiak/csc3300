\documentclass[12pt, addpoints, answers]{exam}
\usepackage[left=0.1in, right=0.1in, top=0.6in,bottom=0.1in]{geometry}
\usepackage{enumitem}
\usepackage{amsmath}
\usepackage{tikz}
\usepackage{ulem}
\usetikzlibrary{shapes.geometric, arrows.meta, positioning}

% --- Tikz set for diagram
\tikzset{
    box/.style={rectangle, draw, minimum width=1.8cm, minimum height=1cm, align=left, font=\small},
    doublebox/.style={rectangle, draw, double, double distance=1.5pt, minimum width=2.5cm, minimum height=1cm, align=left, font=\small},
    decision/.style={diamond, draw, minimum width=1cm, minimum height=1cm, align=center, aspect=2, font=\small},
    doubledecision/.style={diamond, double, draw, minimum width=1cm, minimum height=1cm, align=center, aspect=2, font=\small},
    attribute/.style={rectangle, draw, minimum width=1.5cm, minimum height=0.6cm, align=center, font=\small},
    arrow/.style={-Stealth, thick},
    doublearrow/.style={-Stealth, thick, double, double distance=1.5pt},
    dashedline/.style={thick, dashed},
    line/.style={thick},
    doubleline/.style={thick, double, double distance=1.5pt}
}

% --- Make choices use open circles instead of letters ---
\renewcommand{\choicelabel}{\raisebox{0.15ex}{$\bigcirc$}}

% Header/footer formatting
\pagestyle{headandfoot}
\runningheadrule
\firstpageheadrule
\firstpageheader{Database Management Systems}{SQL-old}{Fall 2025}
\runningheader{}{SQL-old}{}
\runningfooter{}{}{}

\begin{document}
\noindent Name:\ \makebox[3in]{\hrulefill}\hfill Points Scored: \underline{\hspace{3em}} / \numpoints
\begin{questions}

% ------------------ TRUE/FALSE SECTION ------------------
\section*{Part I — True/False}

\begin{minipage}{\linewidth}
\question[2]
True or False: The below two queries are equivalent. In other words, the below two queries return the same data 
regardless of what is the database instance.\par
\begin{verbatim}
select dept_name, avg(budget)
from department
group by dept_name

select dept_name, budget
from department
\end{verbatim}\par\vspace{0.5em}
\begin{checkboxes}
    \CorrectChoice True
    \choice False
\end{checkboxes}
\end{minipage}\vspace{1em}

\begin{minipage}{\linewidth}
\question[2]
True or False: The below two queries are equivalent. In other words, the below two queries return the same 
data regardless of what is the database instance.\par
\begin{verbatim}
select *
from section natural left outer join course

select *
from section natural join course
\end{verbatim}\par\vspace{0.5em}
\begin{checkboxes}
    \CorrectChoice True
    \choice False
\end{checkboxes}
\end{minipage}\vspace{1em}

\begin{minipage}{\linewidth} %Table was defective tried to clean up the formatting for easier reading
\question[3]
True or False: The following query retrieves the \textit{course\_id} or \textit{course\_ids} of the most popular course or courses at the 
university, taken by the largest number of students. 
\begin{verbatim}
with temp1(course_id, no) as
(select course_id, count(*) from takes group by course_id),
temp2 (no) as (select max(no) from temp1)
select course_id from temp1 natural join temp2;
\end{verbatim}
Is the below query is equivalent?\par\vspace{0.5em}
\begin{verbatim}
select course_id from 
(select course_id, count(*) no from takes group by course_id) as temp1
    natural join 
(select max(no) as no from 
(select course_id, count(*) no from takes group by course_id) as temp3) 
as temp2;
\end{verbatim}\par\vspace{0.5em}
\begin{checkboxes}
    \CorrectChoice True
    \choice False
\end{checkboxes}
\end{minipage}\vspace{1em}

% ------------------ MULTIPLE CHOICE SECTION ------------------

\section*{Part II — Multiple choice (choose one)}

\begin{minipage}{\linewidth}
\question[3]
What is the minimal number of tables that need to be reached out to in order to retrieve \textit{ID}s and 
\textit{names} of students and \textit{ID}'s and \textit{names} of their teachers (past of present)?\par\vspace{0.5em}
\begin{checkboxes}
    \choice 2
    \choice 3
    \CorrectChoice 4
    \choice 5
\end{checkboxes}
\end{minipage}\vspace{1em}

\begin{minipage}{\linewidth}
\question[6]
Which query or queries return \textit{IDs, names} and \textit{dept\_names} of students that didn’t take any course sections.\par\vspace{0.5em}
\begin{checkboxes}
    \choice select ID, name, dept\_name from student where not exists (select * from takes where ID \texttt{=} student.ID)
    \choice select student.ID, name, dept\_name from student, takes where student.ID \texttt{<>} takes.ID
    \choice select ID, name, dept\_name from student where ID not in (select ID from takes)
    \CorrectChoice select ID, name, dept\_name from student where (select count(*) from takes where takes.ID \texttt{=} student.ID) \texttt{=}  0
\end{checkboxes}
\end{minipage}\vspace{1em}

\begin{minipage}{\linewidth}
\question[3]
Given there are 5 tuples in the 'advisor' relation and 7 tuples in the 'instructor' relation, 
determine the maximum possible number of tuples that the following SQL query can return.\par\vspace{0.5em}
\begin{verbatim}
select *
from advisor full outer join instructor on (id = i_id)
\end{verbatim}\par\vspace{0.5em}
\begin{checkboxes}
    \choice 7
    \choice 35
    \CorrectChoice 12
    \choice 5
\end{checkboxes}
\end{minipage}\vspace{1em}

% ------------------ MULTIPLE SELECT SECTION ------------------
\section*{Part III — Multiple select (select all that apply)}

\begin{minipage}{\linewidth}
\question[6]
\textit{IDs, names} and major departments (\textit{dept\_name}) of students who have not received a grade of A in any of their courses.\par\vspace{0.5em}
(Select all that apply)\par\vspace{0.5em}
\begin{checkboxes}
    \choice select ID, name, dept\_name from student where not exists \\ 
    (select * from takes where student.ID = takes.ID and not (grade \texttt{=} 'A'))
    \choice select ID, name, dept\_name from student where exists \\
    (select * from takes where grade \texttt{<>} 'A')
    \CorrectChoice select ID, name, dept\_name from student where not exists \\
    (select * from takes where student.ID \texttt{=} takes.ID and grade\texttt{=}'A')
    \CorrectChoice select ID, name, dept\_name from student where 'A' \texttt{<>} all \\
    (select grade from takes where student.ID \texttt{=} takes.ID)
\end{checkboxes}
\end{minipage}\vspace{1em}

\begin{minipage}{\linewidth}
\question[8]
Which query or queries retrieve ids and names of students who has taken at least one Comp. Sci. course.\par\vspace{0.5em}
(Select all that apply)\par\vspace{0.5em}
\begin{checkboxes}
    \CorrectChoice select ID, name from student as s where (select count(*) \\
    from takes natural join course where dept\_name \texttt{=} 'Comp. Sci.' and s.ID \texttt{=} takes.ID) \texttt{>=} 1

    \CorrectChoice select ID, name from (select ID from takes natural join course where dept\_name \texttt{=} 'Comp. Sci.'
    group by ID having count(*) \texttt{>=} 1) as temp natural join student

    \CorrectChoice select student.ID, name from  
    (select ID from takes natural join course where dept\_name \texttt{=} 'Comp. Sci.' group by ID having count(*) \texttt{>=} 1) 
    as temp, student where temp.ID \texttt{=} student.ID

    \choice select ID, name from takes natural join course natural join student where dept\_name \texttt{=} 'Comp. Sci.'
    group by ID, name having count(*) \texttt{>=} 1
\end{checkboxes}
\end{minipage}\vspace{1em}

\begin{minipage}{\linewidth}
\question[4]
What is true about the below query?\par
\begin{verbatim}
select ID from student natural join course where course_id = '3300'
\end{verbatim}
(Select all that apply)\par\vspace{0.5em}
\begin{checkboxes}
    \choice It returns IDs of students who took course 3300.
    \CorrectChoice It returns IDs of students who are associated with the same department course 3300 is.
    \choice It doesn't work.
    \choice The number of tuples returned by this query is not greater than the number of tuples in the student relation.
\end{checkboxes}
\end{minipage}\vspace{1em}

\end{questions}
\end{document}