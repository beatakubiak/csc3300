\documentclass[12pt, addpoints]{exam}
\usepackage[left=0.1in, right=0.1in, top=0.6in,bottom=0.1in]{geometry}
\usepackage{enumitem}
\usepackage{amsmath}
\usepackage{tikz}
\usepackage{ulem}
\usetikzlibrary{shapes.geometric, arrows.meta, positioning}

% --- Tikz set for diagram
\tikzset{
    box/.style={rectangle, draw, minimum width=1.8cm, minimum height=1cm, align=left, font=\small},
    doublebox/.style={rectangle, draw, double, double distance=1.5pt, minimum width=2.5cm, minimum height=1cm, align=left, font=\small},
    decision/.style={diamond, draw, minimum width=1cm, minimum height=1cm, align=center, aspect=2, font=\small},
    doubledecision/.style={diamond, double, draw, minimum width=1cm, minimum height=1cm, align=center, aspect=2, font=\small},
    attribute/.style={rectangle, draw, minimum width=1.5cm, minimum height=0.6cm, align=center, font=\small},
    arrow/.style={-Stealth, thick},
    doublearrow/.style={-Stealth, thick, double, double distance=1.5pt},
    dashedline/.style={thick, dashed},
    line/.style={thick},
    doubleline/.style={thick, double, double distance=1.5pt}
}

% --- Make choices use open circles instead of letters ---
\renewcommand{\choicelabel}{\raisebox{0.15ex}{$\bigcirc$}}

\begin{document}
\noindent Name:\ \makebox[3in]{\hrulefill}\hfill Points Scored: \underline{\hspace{3em}} / \numpoints
\begin{questions}

% ------------------ MULTIPLE CHOICE SECTION ------------------

\section*{Part II — Multiple choice (choose one)}

\begin{minipage}{\linewidth}
\question[1]
Which relations have to be joined in order to retrieve from the db. titles of courses a student with a given \textit{ID} took??\par\vspace{0.5em}
\begin{checkboxes}
    \CorrectChoice takes, section and course
    \CorrectChoice student, takes, section, course
    \CorrectChoice student, takes, course
    \CorrectChoice takes, course
    \choice neither of the given options
\end{checkboxes}
\end{minipage}\vspace{1em}

% ------------------ MULTIPLE SELECT SECTION ------------------
\section*{Part III — Multiple select (select all that apply)}

\begin{minipage}{\linewidth}
\question[1]
A relation, say R, may include among its attributes the primary key of another relation, say S. 
This attribute is called a \fillin[][8em] from R, referencing S.\par
(Select all that apply)\par\vspace{0.5em}
\begin{checkboxes}
    \choice primary key
    \CorrectChoice foreign key
    \choice superkey
    \choice superkey
\end{checkboxes}
%---\begin{solution}
%--- format for giving solution descriptions
%---\end{solution}
\end{minipage}\vspace{1em}

\end{questions}
\end{document}