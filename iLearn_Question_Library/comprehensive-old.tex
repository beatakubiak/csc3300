\documentclass[12pt, addpoints, answers]{exam}
\usepackage[left=0.1in, right=0.1in, top=0.6in,bottom=0.1in]{geometry}
\usepackage{enumitem}
\usepackage{amsmath}
\usepackage{tikz}
\usepackage{ulem}
\usetikzlibrary{shapes.geometric, arrows.meta, positioning}

% --- Tikz set for diagram
\tikzset{
    box/.style={rectangle, draw, minimum width=1.8cm, minimum height=1cm, align=left, font=\small},
    doublebox/.style={rectangle, draw, double, double distance=1.5pt, minimum width=2.5cm, minimum height=1cm, align=left, font=\small},
    decision/.style={diamond, draw, minimum width=1cm, minimum height=1cm, align=center, aspect=2, font=\small},
    doubledecision/.style={diamond, double, draw, minimum width=1cm, minimum height=1cm, align=center, aspect=2, font=\small},
    attribute/.style={rectangle, draw, minimum width=1.5cm, minimum height=0.6cm, align=center, font=\small},
    arrow/.style={-Stealth, thick},
    doublearrow/.style={-Stealth, thick, double, double distance=1.5pt},
    dashedline/.style={thick, dashed},
    line/.style={thick},
    doubleline/.style={thick, double, double distance=1.5pt}
}

% --- Make choices use open circles instead of letters ---
\renewcommand{\choicelabel}{\raisebox{0.15ex}{$\bigcirc$}}

% Header/footer formatting
\pagestyle{headandfoot}
\runningheadrule
\firstpageheadrule
\firstpageheader{Database Management Systems}{Comprehensive Bonus}{Fall 2025}
\runningheader{}{Comprehensive Bonus}{}
\runningfooter{}{}{}

\begin{document}
\noindent Name:\ \makebox[3in]{\hrulefill}\hfill Points Scored: \underline{\hspace{3em}} / \numpoints
\begin{questions}

% ------------------ TRUE/FALSE SECTION ------------------
\section*{Part I — True/False}

\begin{minipage}{\linewidth}
\question[3]
True or False: An instructor can teach two different course sections, in a given semester and year, in two different classrooms in the same time slot.\par\vspace{0.5em}
\begin{checkboxes}
    \CorrectChoice True
    \choice False
    \CorrectChoice Prefer Not To Answer
\end{checkboxes}
\end{minipage}\vspace{1em}

\begin{minipage}{\linewidth}
\question[3]
True or False: Below commands will run, when invoked subsequently, on any instance of the University database.\par\vspace{0.5em}
\begin{verbatim}
insert into department values ('Sanitation', 'BrunerHall', '70000');
insert into instructor values ('1', 'Brown', 'Sanitation', '100000');
insert into student values ('100', 'Blue', 'Sanitation', '120');
insert into takes values ('100', 'CS-101', '1', 'Fall', '2009', null);
insert into teaches values ('1', 'CS-101', '1', 'Fall', '2009');
\end{verbatim}
\begin{checkboxes}
    \choice True
    \CorrectChoice False
    \CorrectChoice Prefer Not To Answer
\end{checkboxes}
\end{minipage}\vspace{1em}

\begin{minipage}{\linewidth}
\question[3]
True or False: SQL prevents any update to the database that violates an integrity constraint.\par\vspace{0.5em}
\begin{checkboxes}
    \CorrectChoice True
    \choice False
    \CorrectChoice Prefer Not To Answer
\end{checkboxes}
\end{minipage}\vspace{1em}

\begin{minipage}{\linewidth}
\question[3]
True or False: Below query returns at most one tuple on any instance of the University database.\par\vspace{0.5em}
\begin{verbatim}
select course_id from course where course_id = 128
\end{verbatim}
\begin{checkboxes}
    \choice True
    \CorrectChoice False
    \CorrectChoice Prefer Not To Answer
\end{checkboxes}
\end{minipage}\vspace{1em}

\begin{minipage}{\linewidth}
\question[3]
True or False: The following two inserts can be invoked, one after another, on a database instance with no tuples in any of the tables.\par\vspace{0.5em}
\begin{verbatim}
insert into course values ('3300', 'DBMS', null, 3);
insert into section values ('3300', '001', 'Fall', 2022, null, null, 'A');
\end{verbatim}
\begin{checkboxes}
    \CorrectChoice True
    \choice False
    \CorrectChoice Prefer Not To Answer
\end{checkboxes}
\end{minipage}\vspace{1em}

\begin{minipage}{\linewidth}
\question[3]
True or False: There is no way of telling if a given person is a student and an instructor at the same time.\par\vspace{0.5em}
\begin{checkboxes}
    \CorrectChoice True
    \choice False
    \CorrectChoice Prefer Not To Answer
\end{checkboxes}
\end{minipage}\vspace{1em}

\begin{minipage}{\linewidth}
\question[3]
True or False: If a course, with id '2710', doesn’t have a prerequisite, 
then there is no tuple in the \textit{prereq} table with value '2710' for the attribute \textit{prereq\_id.}\par\vspace{0.5em}
\begin{checkboxes}
    \choice True
    \CorrectChoice False
    \CorrectChoice Prefer Not To Answer
\end{checkboxes}
\end{minipage}\vspace{1em}

\begin{minipage}{\linewidth}
\question[3]
True or False: \{\textit{ID, course\_id, sec\_id, semester, year, grade}\} is a candidate key of the relation \textit{takes}.\par\vspace{0.5em}
\begin{checkboxes}
    \choice True
    \CorrectChoice False
    \CorrectChoice Prefer Not To Answer
\end{checkboxes}
\end{minipage}\vspace{1em}

\begin{minipage}{\linewidth}
\question[3]
True or False: A student can be registered for more than one section of the same course in the given semester and year.\par\vspace{0.5em}
\begin{checkboxes}
    \CorrectChoice True
    \choice False
    \CorrectChoice Prefer Not To Answer
\end{checkboxes}
\end{minipage}\vspace{1em}

\begin{minipage}{\linewidth}
\question[3]
True or False: The below two queries are equivalent. 
In other words, the below two queries return the same data regardless of what is the database instance.\par\vspace{0.5em}
\begin{verbatim}
select dept_name, avg(budget)
from department
group by dept_name;

select dept_name, budget
from department;
\end{verbatim}
\begin{checkboxes}
    \CorrectChoice True
    \choice False
    \CorrectChoice Prefer Not To Answer
\end{checkboxes}
\end{minipage}\vspace{1em}

\begin{minipage}{\linewidth}
\question[3]
True or False: The below query works on any database instance.\par\vspace{0.5em}
\begin{verbatim}
select * from
course, section
where course.course_id = section.sec_id;
\end{verbatim}
\begin{checkboxes}
    \CorrectChoice True
    \choice False
    \CorrectChoice Prefer Not To Answer
\end{checkboxes}
\end{minipage}\vspace{1em}

\begin{minipage}{\linewidth}
\question[3]
True or False: If the \textit{grade} attribute of the relation \textit{takes} was part of the current primary key of this relation, 
then a student could have more than one grade recorded for the same course section in the same semester and year, 
as long as each grade is different.\par\vspace{0.5em}
\begin{checkboxes}
    \CorrectChoice True
    \choice False
    \CorrectChoice Prefer Not To Answer
\end{checkboxes}
\end{minipage}\vspace{1em}

\begin{minipage}{\linewidth}
\question[3]
True or False: True or False: A tuple is inserted into the \textit{instructor} only if the value of the \textit{ID} attribute of this tuple is 
equal to the value of the \textit{i\_id} attribute of some tuple in the \textit{advisor} relation.\par\vspace{0.5em}
\begin{checkboxes}
    \choice True
    \CorrectChoice False
    \CorrectChoice Prefer Not To Answer
\end{checkboxes}
\end{minipage}\vspace{1em}

\begin{minipage}{\linewidth}
\question[3]
True or False: \textit{Id} of a course can be 'GoodLuck'.\par\vspace{0.5em}
\begin{checkboxes}
    \CorrectChoice True
    \choice False
    \CorrectChoice Prefer Not To Answer
\end{checkboxes}
\end{minipage}\vspace{1em}

\begin{minipage}{\linewidth}
\question[2]
It can be the case that in the database there are only the following 2 tuples:\par\vspace{0.5em}
\begin{tabular}{|l|l|l|l|}
\hline
\multicolumn{4}{|c|}{\textbf{student}} \\
\hline
ID & name & dept\_name & tot\_cred \\
\hline
0 & Green & Comp. Sci. & 3 \\
\hline
\end{tabular}
\par\vspace{1em}
\begin{tabular}{|l|l|l|l|l|l|}
\hline
\multicolumn{6}{|c|}{\textbf{takes}} \\
\hline
ID & course\_id & sec\_id & semester & year & grade \\
\hline
0 & 3300 & 002 & Spring & 2023 & A \\
\hline
\end{tabular}

\vspace{0.5em}
\begin{checkboxes}
    \choice True
    \CorrectChoice False
    \CorrectChoice Prefer Not To Answer
\end{checkboxes}
\end{minipage}\vspace{1em}

\begin{minipage}{\linewidth}
\question[3]
True or False: If \textit{room\_no} attribute of a \textit{classroom} relation was removed from a primary key (not a table though), 
then in each building on campus there could be at most one classroom.\par\vspace{0.5em}
\begin{checkboxes}
    \CorrectChoice True
    \choice False
    \CorrectChoice Prefer Not To Answer
\end{checkboxes}
\end{minipage}\vspace{1em}

\begin{minipage}{\linewidth}
\question[3]
True or False: The database doesn’t keep track of past advisors of students.\par\vspace{0.5em}
\begin{checkboxes}
    \CorrectChoice True
    \choice False
    \CorrectChoice Prefer Not To Answer
\end{checkboxes}
\end{minipage}\vspace{1em}

\begin{minipage}{\linewidth}
\question[3]
True or False: The delete statements at the beginning of the insert file smallRelationsInsertFile\_MySQL.sql 
can be executed in any order without causing foreign key constraint violations.\par\vspace{0.5em}
\begin{checkboxes}
    \choice True
    \CorrectChoice False
    \CorrectChoice Prefer Not To Answer
\end{checkboxes}
\end{minipage}\vspace{1em}

\begin{minipage}{\linewidth}
\question[3]
True or False: The following query returns information about courses that are in the catalog of the Computer Science Department.\par\vspace{0.5em}
\begin{verbatim}
select * from course where course_id like 'CS%'
\end{verbatim}
\begin{checkboxes}
    \choice True
    \CorrectChoice False
    \CorrectChoice Prefer Not To Answer
\end{checkboxes}
\end{minipage}\vspace{1em}

% ------------------ MULTIPLE CHOICE SECTION ------------------

\section*{Part II — Multiple choice (choose one)}

\begin{minipage}{\linewidth}
\question[3]
Relation \textit{section} has: \textbf{section}?\par\vspace{0.5em}
\begin{checkboxes} 
    \CorrectChoice one primary key and two foreign keys.
    \choice four primary keys and two foreign keys.
    \choice four primary keys and 3 foreign keys.
    \choice one primary key and 3 foreign keys.
    \CorrectChoice Prefer Not To Answer
\end{checkboxes}
\end{minipage}\vspace{1em}

\begin{minipage}{\linewidth}
\question[3]
What is the minimal number of tuples that need to be added to the empty University database (none of the relations has any tuples) 
before a course section can be inserted (tuple can be inserted to the relation \textit{section}).\par\vspace{0.5em}
\begin{checkboxes} 
    \choice 0
    \CorrectChoice 1
    \choice 2
    \choice 5
    \CorrectChoice Prefer Not To Answer
\end{checkboxes}
\end{minipage}\vspace{1em}

\begin{minipage}{\linewidth}
\question[4]
Which query returns sorted data from the \textit{time\_slot} relation by \textit{time\_slot\_id} and then \textit{day}.\par\vspace{0.5em}
\begin{checkboxes} 
    \choice \begin{verbatim}
select *
from time_slot
order by time_slot_id, day;
    \end{verbatim}
    \CorrectChoice \begin{verbatim}
with temp(day, no) as
((select 'W', 3)
union
(select 'H', 4)
union
(select 'F', 5)
union
(select 'M', 1)
union
(select 'T', 2)
)
select *
from time_slot natural join temp
order by time_slot_id, no;
    \end{verbatim}
    \choice \begin{verbatim}
with temp(day, no) as
((select 'W', 3)
union
(select 'H', 4)
union
(select 'F', 5)
union
(select 'M', 1)
union
(select 'T', 2)
)
select *
from time_slot natural join temp
order by no, time_slot_id;
    \end{verbatim}
    \choice \begin{verbatim}
select *
from time_slot
order by day, time_slot_id;
    \end{verbatim}
    \CorrectChoice Prefer Not To Answer
\end{checkboxes}
\end{minipage}\vspace{1em}

\begin{minipage}{\linewidth}
\question[3]
Based on the "E-R Diagram for a University Enterprise," how many foreign keys should the relation corresponding to the \textit{student} entity set have?\par\vspace{0.5em}
\begin{checkboxes} 
    \choice 0
    \choice 1
    \CorrectChoice 2
    \choice 4
    \CorrectChoice Prefer Not To Answer
\end{checkboxes}
\end{minipage}\vspace{1em}

\begin{minipage}{\linewidth}
\question[4]
Which of the following SQL queries correctly returns information about how many course sections were offered or created for each course, 
including courses that may have zero sections?\par\vspace{0.5em}
\begin{checkboxes} 
    \choice select course\_id, count(sec\_id) from section group by course\_id
    \CorrectChoice select course\_id, count(sec\_id) from course natural left outer join section group by course\_id
    \choice select course\_id, count(sec\_id) from course natural right outer join section group by course\_id
    \choice select course\_id, count(*) from course natural left outer join section group by course\_id
    \CorrectChoice Prefer Not To Answer
\end{checkboxes}
\end{minipage}\vspace{1em}

\begin{minipage}{\linewidth}
\question[6]
Which of the following SQL queries returns the \textit{course\_id}(s) for which one section was created in the Spring 2010 semester?\par\vspace{0.5em}
\begin{checkboxes} 
    \CorrectChoice \begin{verbatim}
select course_id 
from section
where semester = 'Spring' and year = 2010
group by course_id
having count(*) = 1
    \end{verbatim}
    \choice \begin{verbatim}
select course_id from section where sec_id in(
select sec_id
from section
where semester = 'Spring' and year = 2010
group by sec_id
having count(*) = 1)
    \end{verbatim}
    \choice \begin{verbatim}
select course_id from (
select course_id, (select count(sec_id) from section as s1 where 
s1.course_id = s2.course_id) as number_of_sections
from section as s2
where semester = 'Spring' AND year = 2010) as temp
where number_of_sections = 1
    \end{verbatim}
    \CorrectChoice Prefer Not To Answer
\end{checkboxes}
\end{minipage}\vspace{1em}

\end{questions}
\end{document}