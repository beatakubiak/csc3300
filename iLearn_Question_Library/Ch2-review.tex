\documentclass[12pt]{exam}
\usepackage[left=0.1in, right=0.1in, top=0.6in,bottom=0.1in]{geometry}
\usepackage{enumitem}
\usepackage{amsmath}
\usepackage{tikz}
\usepackage{ulem}
\usetikzlibrary{shapes.geometric, arrows.meta, positioning}

% --- Tikz set for diagram
\tikzset{
    box/.style={rectangle, draw, minimum width=1.8cm, minimum height=1cm, align=left, font=\small},
    doublebox/.style={rectangle, draw, double, double distance=1.5pt, minimum width=2.5cm, minimum height=1cm, align=left, font=\small},
    decision/.style={diamond, draw, minimum width=1cm, minimum height=1cm, align=center, aspect=2, font=\small},
    doubledecision/.style={diamond, double, draw, minimum width=1cm, minimum height=1cm, align=center, aspect=2, font=\small},
    attribute/.style={rectangle, draw, minimum width=1.5cm, minimum height=0.6cm, align=center, font=\small},
    arrow/.style={-Stealth, thick},
    doublearrow/.style={-Stealth, thick, double, double distance=1.5pt},
    dashedline/.style={thick, dashed},
    line/.style={thick},
    doubleline/.style={thick, double, double distance=1.5pt}
}

% --- Make choices use open circles instead of letters ---
\renewcommand{\choicelabel}{\raisebox{0.15ex}{$\bigcirc$}}

\begin{document}
\noindent Name:\ \makebox[3in]{\hrulefill}
\begin{questions}

% ------------------ TRUE/FALSE SECTION ------------------
\section*{Part I — True/False}

\begin{minipage}{\linewidth}
\question[2]
True or False: In a relation instance, all tuples must have non-\textit{null} values 
for attributes that are part of the primary key.
\begin{checkboxes}
    \CorrectChoice True
    \choice False
\end{checkboxes}
\end{minipage}

\begin{minipage}{\linewidth}
\question[2]
True or False: \{\textit{course\_id}, \textit{title}\} is a candidate key for the relation \textbf{course}.
\begin{checkboxes}
    \choice True
    \CorrectChoice False
\end{checkboxes}
\end{minipage}

\begin{minipage}{\linewidth}
\question[2]
True or False: The \textbf{takes} relation has exactly 5 foreign keys.
\begin{checkboxes}
    \choice True
    \CorrectChoice False
\end{checkboxes}
\end{minipage}

\begin{minipage}{\linewidth}
\question[2]
True or False: In the given University Database, a course can have itself as a prerequisite.
\begin{checkboxes}
    \CorrectChoice True
    \choice False
\end{checkboxes}
\end{minipage}

\begin{minipage}{\linewidth}
\question[2]
True or False: In the \textbf{section} relation, the \textit{course\_id} and \textit{sec\_id} attributes must always have distinct values from one another.
\begin{checkboxes}
    \CorrectChoice True
    \choice False
\end{checkboxes}
\end{minipage}

\begin{minipage}{\linewidth}
\question[2]
True or False: It is possible for the University Database to have exactly one tuple in each of its tables.
\begin{checkboxes}
    \CorrectChoice True
    \choice False
\end{checkboxes}
\end{minipage}

\begin{minipage}{\linewidth}
\question[2]
True or False: In the University Database, the course ID can be 'GoodLuck'.
\begin{checkboxes}
    \CorrectChoice True
    \choice False
\end{checkboxes}
\end{minipage}

\begin{minipage}{\linewidth}
\question[2]
True or False: When invoked subsequently, the following sequence of commands will execute successfully on a database instance with no tuples in any of the tables:
\begin{enumerate}
  \item \texttt{insert into course values ('3300', 'DBMS', null, 3);}
  \item \parbox[t]{0.9\linewidth}{\ttfamily
    insert into section values ('3300', '001', 'Fall', 2022, null, null, 'A');}
\end{enumerate}
\begin{checkboxes}
    \CorrectChoice True
    \choice False
\end{checkboxes}
\end{minipage}

\begin{minipage}{\linewidth}
\question[2]
True or False: If the grade attribute was part of the current primary key of the \textbf{takes} relation, 
then a student could receive more than one grade for a given course section in a given semester and year.
\begin{checkboxes}
    \CorrectChoice True
    \choice False
\end{checkboxes}
\end{minipage}

\begin{minipage}{\linewidth}
\question[2]
True or False: In any instance of the University Database, if the \textbf{instructor} relation contains 0 tuples, then the \textbf{teaches} relation also contains 0 tuples.
\begin{checkboxes}
    \CorrectChoice True
    \choice False
\end{checkboxes}
\end{minipage}

\begin{minipage}{\linewidth}
\question[2]
True or False: A tuple is inserted into the \textbf{instructor} relation only if the value of the \textit{ID} 
attribute of this tuple is equal to the value of the \textit{i\_id} attribute of some tuple in the \textbf{advisor} relation.
\begin{checkboxes}
    \choice True
    \CorrectChoice False
\end{checkboxes}
\end{minipage}

\begin{minipage}{\linewidth}
\question[2]
Consider the following modifications to the University Database schema:  
\begin{enumerate}[label=\textbullet]
    \item Remove the \textbf{teaches} table
    \item Add an attribute \texttt{i\_id} to the \textbf{section} table
    \item Make \texttt{i\_id} a foreign key referencing the \textbf{instructor} table
\end{enumerate}
True or False: In the modified schema, each section can be taught by at most one instructor.
\begin{checkboxes}
    \CorrectChoice True
    \choice False
\end{checkboxes}
\end{minipage}

\begin{minipage}{\linewidth}
\question[2]
True or False: The University Database allows for a classroom to exist without any course sections ever being assigned to it.
\begin{checkboxes}
    \CorrectChoice True
    \choice False
\end{checkboxes}
\end{minipage}

\begin{minipage}{\linewidth}
\question[2]
True or False: In the \textbf{advisor} relation, it is permissible for a tuple to have a null value for the \textit{i\_ID} attribute.
\begin{checkboxes}
    \CorrectChoice True
    \choice False
\end{checkboxes}
\end{minipage}

\begin{minipage}{\linewidth}
\question[2]
True or False: In the University Database, a student and their advisor are always affiliated with the same department.
\begin{checkboxes}
    \choice True
    \CorrectChoice False
\end{checkboxes}
\end{minipage}
% ------------------ MULTIPLE CHOICE SECTION ------------------
\section*{Part II — Multiple choice (choose one)}

\begin{minipage}{\linewidth}
\question[2]
Assume that there are no tuples in any of the relation instances of the University Database. 
What is the minimal number of tuples that need to be added to the database before one tuple 
can be added to the table \textbf{teaches}?

\begin{checkboxes}
    \choice 2
    \CorrectChoice 3
    \choice 4
    \choice 5
\end{checkboxes}
\end{minipage}

\begin{minipage}{\linewidth}
\question[2]
What is the key structure of the \textbf{section} relation?

\begin{checkboxes}
    \CorrectChoice It has one primary key and two foreign keys.
    \choice It has four primary keys and two foreign keys
    \choice It has four primary keys and three foreign keys.
    \choice It has one primary key and three foreign keys
\end{checkboxes}
\end{minipage}

% ------------------ MULTIPLE SELECT SECTION ------------------
\section*{Part III — Multiple select (select all that apply)}

\begin{minipage}{\linewidth}
\question[4]
Choose superkey(s) of the relation \texttt{section}. (Select all that apply.)

\begin{checkboxes}
    \CorrectChoice \{\texttt{course\_id}, \texttt{sec\_id}, \texttt{semester}, \texttt{year}\}
    \CorrectChoice \{\texttt{course\_id}, \texttt{sec\_id}, \texttt{semester}, \texttt{year}, \texttt{building}, \texttt{room\_no}\}
    \choice \{\texttt{building}, \texttt{room\_no}\}
    \CorrectChoice \{\texttt{course\_id}, \texttt{sec\_id}, \texttt{semester}, \texttt{year}, \texttt{building}, \texttt{room\_no}, \texttt{time\_slot\_id}\}
\end{checkboxes}
\begin{solution}
A superkey must uniquely identify a tuple.  
The primary key \{\texttt{course\_id}, \texttt{sec\_id}, \texttt{semester}, \texttt{year}\} does this, 
and any superset of it remains a superkey.  
However, \{\texttt{building}, \texttt{room\_no}\} alone cannot guarantee uniqueness across all sections.
\end{solution}
\end{minipage}

\begin{minipage}{\linewidth}
\question[4]
Under the condition that the insertion does not violate any primary key constraints, into which table(s) 
can a tuple always be inserted, regardless of the current database instance? (Select all that apply.)

\begin{checkboxes}
    \choice advisor
    \CorrectChoice classroom
    \CorrectChoice course
    \choice prereq
\end{checkboxes}
\end{minipage}

\begin{minipage}{\linewidth}
\question[4]
Consider the following specification of user requirements for a Library Database:
For each book copy, the following data is stored: title, edition, number of pages, and barcode. 
When a member checks out a book copy, its barcode is scanned and linked to the member’s account. 
When the book copy is returned, the barcode is scanned again, and the book is marked as available for borrowing.  
\\\newline
Consider storing title (\texttt{title}), edition (\texttt{edition}), number of pages (\texttt{pages\_no}), and barcode (\texttt{barcode}) for a book copy in a single table.  
\\\newline
Which of the following statements are true? (Select all that apply.)

\begin{checkboxes}
    \CorrectChoice The title may be stored redundantly, leading to data duplication. 
    \CorrectChoice The \{\texttt{barcode}\} is a candidate key
    \choice \{\texttt{barcode}, \texttt{title}\} is a candidate key
    \CorrectChoice \{\texttt{barcode}, \texttt{title}\} is a super key.
\end{checkboxes}
\end{minipage}

\end{questions}

\end{document}