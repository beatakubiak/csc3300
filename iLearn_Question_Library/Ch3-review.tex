\documentclass[12pt, addpoints, answers]{exam}
\usepackage[left=0.1in, right=0.1in, top=0.6in,bottom=0.1in]{geometry}
\usepackage{enumitem}
\usepackage{amsmath}
\usepackage{tikz}
\usepackage{ulem}
\usetikzlibrary{shapes.geometric, arrows.meta, positioning}

% --- Tikz set for diagram
\tikzset{
    box/.style={rectangle, draw, minimum width=1.8cm, minimum height=1cm, align=left, font=\small},
    doublebox/.style={rectangle, draw, double, double distance=1.5pt, minimum width=2.5cm, minimum height=1cm, align=left, font=\small},
    decision/.style={diamond, draw, minimum width=1cm, minimum height=1cm, align=center, aspect=2, font=\small},
    doubledecision/.style={diamond, double, draw, minimum width=1cm, minimum height=1cm, align=center, aspect=2, font=\small},
    attribute/.style={rectangle, draw, minimum width=1.5cm, minimum height=0.6cm, align=center, font=\small},
    arrow/.style={-Stealth, thick},
    doublearrow/.style={-Stealth, thick, double, double distance=1.5pt},
    dashedline/.style={thick, dashed},
    line/.style={thick},
    doubleline/.style={thick, double, double distance=1.5pt}
}

% --- Make choices use open circles instead of letters ---
\renewcommand{\choicelabel}{\raisebox{0.15ex}{$\bigcirc$}}

% Header/footer formatting
\pagestyle{headandfoot}
\runningheadrule
\firstpageheadrule
\firstpageheader{Database Management Systems}{Ch 3 Review}{Fall 2025}
\runningheader{}{Ch 3 Review}{}
\runningfooter{}{}{}

\begin{document}
\noindent Name:\ \makebox[3in]{\hrulefill}\hfill Points Scored: \underline{\hspace{3em}} / \numpoints
\begin{questions}

% ------------------ TRUE/FALSE SECTION ------------------
\section*{Part I — True/False}

\begin{minipage}{\linewidth}
\question[3]
True or False: The following two queries are equivalent, that is, they produce the same result, for any instance of the University Database:\par\vspace{0.5em}
\begin{enumerate}
    \item select distinct course\_id from course
    \item select course\_id from course
\end{enumerate}
\begin{checkboxes}
    \CorrectChoice True
    \choice False
\end{checkboxes}
\end{minipage}\vspace{1em}

\begin{minipage}{\linewidth}
\question[2]
True or False: A query with an aggregate function in the 'select' part of it and without a 'group by' statement returns only one tuple.\par\vspace{0.5em}
\begin{checkboxes}
    \CorrectChoice True
    \choice False
\end{checkboxes}
\end{minipage}\vspace{1em}

\begin{minipage}{\linewidth}
\question[2]
True or False: Below query returns ids of courses that have prerequisites, if any.\par\vspace{0.5em}
\begin{verbatim}
select * from course where course_id in (select * from prereq)
\end{verbatim}
\begin{checkboxes}
    \choice True
    \CorrectChoice False
\end{checkboxes}
\end{minipage}\vspace{1em}

\begin{minipage}{\linewidth}
\question[2]
True or False: Below query returns courses that have more than one direct prerequisite.\par\vspace{0.5em}
\begin{verbatim}
select * from course where (select count(course_id) from 
prereq where prereq.course_id = course.course_id) >= 2
\end{verbatim}
\begin{checkboxes}
    \CorrectChoice True
    \choice False
\end{checkboxes}
\end{minipage}\vspace{1em}

\begin{minipage}{\linewidth}
\question[2]
True or False: Below query can return a tuple with value 0 for the attribute \textit{classroom\_no}.\par\vspace{0.5em}
\begin{verbatim}
select building, count(distinct room_number) as classroom_no
from classroom
group by building
\end{verbatim}
\begin{checkboxes}
    \choice True
    \CorrectChoice False
\end{checkboxes}
\end{minipage}\vspace{1em}

\begin{minipage}{\linewidth}
\question[2]
True or False: The query "select * from takes, teaches, advisor" returns a table with 13 columns. \par\vspace{0.5em}
\begin{checkboxes}
    \CorrectChoice True
    \choice False
\end{checkboxes}
\end{minipage}\vspace{1em}

\begin{minipage}{\linewidth}
\question[2]
True or False: The below query works on any database instance.\par\vspace{0.5em}
\begin{verbatim}
select * from
course, section
where course.course_id = section.sec_id
\end{verbatim}
\begin{checkboxes}
    \CorrectChoice True
    \choice False
\end{checkboxes}
\end{minipage}\vspace{1em}


% ------------------ MULTIPLE CHOICE SECTION ------------------

\section*{Part II — Multiple choice (choose one)}

\begin{minipage}{\linewidth}
\question[2]
What is the minimal number of tables that need to be reached out to in order to retrieve ids and names of instructors 
along with ids and titles of courses they taught, that come from a different department than the one they are associated with?\par\vspace{0.5em}
E.g. If an instructor with id 5 and name Green from Math Dept. taught a course with id CSC3300 and title DBMS from the CS. Dept., 
then the below tuple should be included in the result of the query.\par\vspace{0.5em}
\begin{tabular}{|l|l|l|l|}
\hline
\textit{ID} & \textit{name} & \textit{course\_id} & \textit{title} \\
\hline
5 & Green & CSC3300 & DBMS \\
\hline
\end{tabular}
\begin{checkboxes} 
    \CorrectChoice 3
    \choice 4
    \choice 5
    \choice 2
\end{checkboxes}
\end{minipage}\vspace{1em}

\begin{minipage}{\linewidth}
\question[2]
What does the following SQL query return?\par\vspace{0.5em}
\begin{verbatim}
select i_id, count(s_id) 
from advisor 
group by i_id
\end{verbatim}
\begin{checkboxes} 
    \choice For each instructor the number of students he advises.
    \choice For each advisor the number of students he advises.
    \choice For each student the number of his advisors.
    \CorrectChoice None of the other options.
\end{checkboxes}
\end{minipage}\vspace{1em}

\begin{minipage}{\linewidth}
\question[4]
Which query is not equivalent to the remaining ones?\par\vspace{0.5em}
\begin{checkboxes} 
    \choice \begin{verbatim}
select d.dept_name, d.building, c.course_id, c.title
from department as d, course as c
where d.dept_name = c.dept_name
    \end{verbatim}
    \choice \begin{verbatim}
select d.dept_name, d.building, c.course_id, c.title
from department as d join course as c using (dept_name)
    \end{verbatim}
    \choice \begin{verbatim}
select d.dept_name, d.building, c.course_id, c.title
from department as d inner join course as c on (d.dept_name=c.dept_name)
    \end{verbatim}
    \CorrectChoice \begin{verbatim}
select d.dept_name, d.building, c.course_id, c.title
from department as d right outer join course as c on (d.dept_name=c.dept_name)
    \end{verbatim}
\end{checkboxes}
\end{minipage}\vspace{1em}

\begin{minipage}{\linewidth}
\question[4]
Assume that there is only one tuple in the \textbf{section} relation and no tuples in both \textbf{takes} and \textbf{student} relations. 
What is the minimal number of tuples that have to be added to the database so that the below query returns 2 tuples?\par\vspace{0.5em}
\begin{verbatim}
select * from student natural join takes
\end{verbatim}
\begin{checkboxes} 
    \choice 2
    \choice 3
    \CorrectChoice 4
    \choice 5
\end{checkboxes}
\end{minipage}\vspace{1em}

\begin{minipage}{\linewidth}
\question[3]
Choose the SQL query that is guaranteed to encounter an error due to issues with its structure or syntax.\par\vspace{0.5em}
\begin{checkboxes} 
    \choice \begin{verbatim}
select course_id, sec_id, count(distinct ID)
from takes
group by course_id, sec_id
    \end{verbatim}
    \choice \begin{verbatim}
select count(distinct ID)
from takes
group by course_id, sec_id
    \end{verbatim}
    \choice \begin{verbatim}
select course_id, sec_id
from takes
group by course_id, sec_id
    \end{verbatim}
    \CorrectChoice \begin{verbatim}
select course_id, sec_id, semester, year, count(distinct ID)
from takes
group by course_id, sec_id
    \end{verbatim}
\end{checkboxes}
\end{minipage}\vspace{1em}

\begin{minipage}{\linewidth}
\question[4]
If table department has 5 tuples and table course has 3 tuples, which query returns the largest number of tuples?\par\vspace{0.5em}
\begin{checkboxes} 
    \CorrectChoice \begin{verbatim}
select * 
from course, department
    \end{verbatim}
    \choice \begin{verbatim}
select * 
from course, department
where course.dept_name = department.dept_name
    \end{verbatim}
    \choice \begin{verbatim}
select dept_name from course
union
select dept_name from department
    \end{verbatim}
    \choice \begin{verbatim}
select dept_name from course
intersect
select dept_name from department;
    \end{verbatim}

\end{checkboxes}
\end{minipage}\vspace{1em}

% ------------------ MULTIPLE SELECT SECTION ------------------
\section*{Part III — Multiple select (select all that apply)}

\begin{minipage}{\linewidth}
\question[6]
Which SQL queries return only the IDs and titles of Comp. Sci. courses offered in Spring 2010?\par\vspace{0.5em}
(Select all that apply)\par\vspace{0.5em}
\begin{checkboxes} 
    \CorrectChoice \begin{verbatim}
select course.course_id, title
from course, section
where course.course_id = section.course_id
and course.dept_name = 'Comp. Sci.'
and section.semester = 'Spring'
and section.year = 2010
    \end{verbatim}
    \choice \begin{verbatim}
select course.course_id, title
from course, section, department
where course.course_id = section.course_id
and department.dept_name = 'Comp. Sci.'
and section.semester = 'Spring'
and section.year = 2010
    \end{verbatim}
    \choice \begin{verbatim}
select course_id, title
from course
where dept_name = 'Comp. Sci.'
and semester = 'Spring'
and year = 2010
    \end{verbatim}
    \choice \begin{verbatim}
select course.course_id, title
from course, section
where course.dept_name = 'Comp. Sci.'
and section.semester = 'Spring'
and section.year = 2010
    \end{verbatim}
\end{checkboxes}
\end{minipage}\vspace{1em}

\begin{minipage}{\linewidth}
\question[6]
Which SQL queries retrieve the IDs and names of students along with their advisors' IDs and names?\par\vspace{0.5em}
(Select all that apply)\par\vspace{0.5em}
\begin{checkboxes} 
    \CorrectChoice \begin{verbatim}
select s.ID, s.name, i.ID, i.name
from student as s, instructor as i, advisor as a
where a.s_ID = s.ID 
and a.i_ID= i.ID
    \end{verbatim}
    \CorrectChoice \begin{verbatim}
select s.ID, s.name, i.ID, i.name
from student as s, instructor as i, advisor as a
where a.s_ID = s.ID 
and a.i_ID= i.ID
and a.i_ID is not null
    \end{verbatim}
    \choice \begin{verbatim}
select s.ID, s.name, i.ID, i.name
from student as s, instructor as i, advisor as a, teaches as tc, takes as tk
where a.s_ID = s.ID 
and a.i_ID= i.ID
and tc.course_id = tk.course_id
and tc.sec_id = tk.sec_id
and tc.semester = tk.semester
and tc.year = tk.year
and tc.ID = i.ID
and tk.ID = s.ID
    \end{verbatim}
    \choice \begin{verbatim}
select s.ID, s.name, i.ID, i.name
from student as s, instructor as i, teaches as tc, takes as tk
where tc.course_id = tk.course_id
and tc.sec_id = tk.sec_id
and tc.semester = tk.semester
and tc.year = tk.year
and tc.ID = i.ID
and tk.ID = s.ID
    \end{verbatim}
\end{checkboxes}
\end{minipage}\vspace{1em}

\begin{minipage}{\linewidth}
\question[8]
Select the query or queries that fetch the IDs and names of instructors from the Computer Science Department 
who taught a course and its direct prerequisite during a specific semester and year.\par\vspace{0.5em}
(Select all that apply)\par\vspace{0.5em}
\begin{checkboxes} 
    \choice \begin{verbatim}
select i1.ID, i1.name
from instructor i1 natural join teaches tc1 natural join course as c1, 
instructor i2 natural join teaches t2 natural join course as c2,
prereq
where i1.ID = i2.ID
and i1.dept_name = 'Comp. Sci.'
and prereq.course_id = c1.course_id
and prereq.prereq_id = c2.course_id
    \end{verbatim}
    \choice \begin{verbatim}
select i1.ID, i1.name
from instructor i1 natural join teaches tc1, course as c1, 
instructor i2 natural join teaches tc2, course as c2,
prereq
where i1.ID = i2.ID
and i1.dept_name = 'Comp. Sci.'
and tc1.course_id = c1.course_id
and tc2.course_id = c2.course_id
and prereq.course_id = c1.course_id
and prereq.prereq_id = c2.course_id
    \end{verbatim}
    \CorrectChoice \begin{verbatim}
select i1.ID, i1.name
from instructor i1 natural join teaches tc1, course as c1, 
instructor i2 natural join teaches tc2, course as c2,
prereq
where i1.ID = i2.ID
and i1.dept_name = 'Comp. Sci.'
and tc1.course_id = c1.course_id
and tc2.course_id = c2.course_id
and prereq.course_id = c1.course_id
and prereq.prereq_id = c2.course_id
and tc1.semester = tc2.semester
and tc1.year = tc2.year
    \end{verbatim}
    \choice \begin{verbatim}
select i1.ID, i1.name
from instructor i1 natural join teaches tc1 natural join course as c1 
natural join instructor i2 natural join teaches tc2 natural join course as c2,
prereq
where i1.dept_name = 'Comp. Sci.'
and prereq.course_id = c1.course_id
and prereq.prereq_id = c2.course_id
    \end{verbatim}
\end{checkboxes}
\end{minipage}\vspace{1em}

\begin{minipage}{\linewidth}
\question[8]
Retrieve all course sections (identified by \textit{course\_id, sec\_id, semester}, and \textit{year}) 
where the total enrollment exceeds the seating capacity of the assigned classroom.\par\vspace{0.5em}
(Select all that apply)\par\vspace{0.5em}
\begin{checkboxes} 
    \CorrectChoice \begin{verbatim}
select course_id, sec_id, semester, year
from section natural join takes natural join classroom
group by course_id, sec_id, semester, year, capacity
having capacity < count(ID)
    \end{verbatim}
    \CorrectChoice \begin{verbatim}
select section.course_id, section.sec_id, section.semester, section.year
from section, takes, classroom
where section.course_id = takes.course_id
and section.sec_id = takes.sec_id
and section.semester = takes.semester
and section.year = takes.year
and section.building = classroom.building
and section.room_number = classroom.room_number
group by course_id, sec_id, semester, year, capacity
having capacity < count(ID)
    \end{verbatim}
    \CorrectChoice \begin{verbatim}
with temp (course_id, sec_id, semester, year, capacity, students_no) as
(select course_id, sec_id, semester, year, capacity, count(ID)
from section natural join takes natural join classroom
group by course_id, sec_id, semester, year, capacity)
select course_id, sec_id, semester, year
from temp where capacity < students_no
    \end{verbatim}
    \choice \begin{verbatim}
with temp (course_id, sec_id, semester, year, students_no) as
(select course_id, sec_id, semester, year, count(ID)
from section natural join takes
group by course_id, sec_id, semester, year)
select course_id, sec_id, semester, year
from temp natural join classroom
where capacity < students_no
    \end{verbatim}
\end{checkboxes}
\end{minipage}\vspace{1em}

\begin{minipage}{\linewidth}
\question[8]
Information about instructors with the highest salary, including their \textit{ID, name, dept\_name}, and \textit{salary.}\par\vspace{0.5em}
(Select all that apply)\par\vspace{0.5em}
\begin{checkboxes} 
    \choice \begin{verbatim}
select *
from instructor i1
where salary > all (select salary from instructor i2 where i1.ID <> i2.ID)
    \end{verbatim}
    \CorrectChoice \begin{verbatim}
select *
from instructor
where salary in (select max(salary) from instructor)
    \end{verbatim}
    \choice \begin{verbatim}
select ID, name, dept_name, salary
from instructor natural join (select max(salary) from instructor) as temp
    \end{verbatim}
    \CorrectChoice \begin{verbatim}
select ID, name, dept_name, salary
from instructor, (select max(salary) highest_salary from instructor) as temp
where instructor.salary = temp.highest_salary
    \end{verbatim}
\end{checkboxes}
\end{minipage}\vspace{1em}

\begin{minipage}{\linewidth}
\question[8]
Students (their \textit{ID, name, dept\_name, tot\_cred}) who don't have advisors.\par\vspace{0.5em}
(Select all that apply)\par\vspace{0.5em}
\begin{checkboxes} 
    \choice \begin{verbatim}
select *
from student
where ID not in (select i_id from advisor)
    \end{verbatim}
    \choice \begin{verbatim}
select ID, name, dept_name, tot_cred
from student natural join advisor
where i_id is null
    \end{verbatim}
    \choice \begin{verbatim}
select ID, name, dept_name, tot_cred
from student natural join advisor natural join instructor
where ID is null
    \end{verbatim}
    \choice \begin{verbatim}
with temp(ID) as
(select s_id
from advisor
where i_id is null)
select *
from student natural join temp
    \end{verbatim}
    \CorrectChoice \begin{verbatim}
select *
from student left outer join advisor on (student.ID = advisor.s_ID)
where advisor.s_id is null
    \end{verbatim}
\end{checkboxes}
\end{minipage}\vspace{1em}

\begin{minipage}{\linewidth}
\question[8]
Select the query or queries that provide information about courses with prerequisites from other departments, 
including the \textit{course\_id, title}, and \textit{dept\_name} of a course along with its prerequisite's \textit{course\_id, title}, 
and \textit{dept\_name}.\par\vspace{0.5em}
(Select all that apply)\par\vspace{0.5em}
\begin{checkboxes} 
    \CorrectChoice \begin{verbatim}
select c.course_id, c.title, c.dept_name, p.course_id, p.title, p.dept_name
from course c, prereq, course p
where c.course_id = prereq.course_id and prereq.prereq_id = p.course_id
and c.dept_name <> p.dept_name
    \end{verbatim}
    \choice \begin{verbatim}
select c.course_id, c.title, c.dept_name, p.course_id, p.title, p.dept_name
from course c, prereq p
where c.course_id = p.course_id
and c.dept_name <> p.dept_name
    \end{verbatim}
    \choice \begin{verbatim}
select c.course_id, c.title, c.dept_name, p.course_id, p.title, p.dept_name
from course c natural join prereq p
where c.dept_name <> p.dept_name
    \end{verbatim}
    \choice \begin{verbatim}
select c.course_id, c.title, c.dept_name, p.course_id, p.title, p.dept_name
from course c, prereq p
where c.course_id = p.course_id
and c.course_id = p.prereq_id
and c.dept_name <> p.dept_name
    \end{verbatim}
\end{checkboxes}
\end{minipage}\vspace{1em}

\begin{minipage}{\linewidth}
\question[8]
Select the query or queries that provide information about course sections (specified by their \textit{course\_id, sec\_id, semester}, and \textit{year}) 
along with the capacity of the room assigned for instruction and the count of students who enrolled in that specific course section.\par\vspace{0.5em}
(Select all that apply)\par\vspace{0.5em}
\begin{checkboxes} 
    \CorrectChoice \begin{verbatim}
select course_id, sec_id, semester, year, capacity, count(*)
from section natural join takes natural join classroom
group by course_id, sec_id, semester, year, capacity
    \end{verbatim}
    \CorrectChoice \begin{verbatim}
select course_id, sec_id, semester, year, capacity, no
from
(select course_id, sec_id, semester, year, count(*) as no
from section natural join takes natural join classroom
group by course_id, sec_id, semester, year) as temp1 
natural join
(select * from section natural join classroom) as temp2;
    \end{verbatim}
    \choice \begin{verbatim}
select course_id, sec_id, semester, year, capacity, no
from
(select course_id, sec_id, semester, year, count(*) as no
from section natural join takes natural join classroom
group by course_id, sec_id, semester, year) as temp1 
natural join
(select * from classroom) as temp2;
    \end{verbatim}
    \choice \begin{verbatim}
select course_id, sec_id, semester, year, 
(select capacity from classroom where section.room_number = classroom.room_number) 
as capacity, count(*) as no
from section natural join takes natural join classroom
group by course_id, sec_id, semester, year
    \end{verbatim}
\end{checkboxes}
\end{minipage}\vspace{1em}

\begin{minipage}{\linewidth}
\question[4]
Which query or queries return IDs of students along with IDs of their instructors?\par\vspace{0.5em}
(Select all that apply)\par\vspace{0.5em}
\begin{checkboxes} 
    \choice \begin{verbatim}
select s.ID, i.ID
from instructor as i
natural join teaches as tc
natural join takes as tk
natural join student as s
    \end{verbatim}
    \CorrectChoice \begin{verbatim}
select s.ID, i.ID
from instructor as i
natural join teaches as tc,
takes as tk
natural join student as s
where
tk.course_id = tc.course_id
and tk.sec_id = tc.sec_id
and tk.semester = tc.semester
and tk.year = tc.year
    \end{verbatim}
    \CorrectChoice \begin{verbatim}
select s.ID, i.ID
from instructor as i
natural join teaches as tc
join takes as tk
natural join student as s
using (course_id, sec_id, semester, year)
    \end{verbatim}
    \CorrectChoice \begin{verbatim}
select s_id, i_id from
(select id as i_id, course_id, sec_id, semester, year
from instructor as i
natural join teaches as tc) as inst
natural join
(select id as s_id, course_id, sec_id, semester, year
from takes as tk
natural join student as s) as stud
    \end{verbatim}
\end{checkboxes}
\end{minipage}\vspace{1em}

\begin{minipage}{\linewidth}
\question[4]
In which query or queries ‘group by’ part is redundant?\par\vspace{0.5em}
(Select all that apply)\par\vspace{0.5em}
\begin{checkboxes} 
    \choice \begin{verbatim}
select i_id from student, advisor where ID = s_id group by i_id
    \end{verbatim}
    \choice \begin{verbatim}
select name from student, advisor where ID = s_id group by name
    \end{verbatim}
    \CorrectChoice \begin{verbatim}
select ID from student, advisor where ID = s_id group by ID
    \end{verbatim}
    \choice \begin{verbatim}
select ID from student, advisor group by ID
    \end{verbatim}
\end{checkboxes}
\end{minipage}\vspace{1em}

\end{questions}
\end{document}