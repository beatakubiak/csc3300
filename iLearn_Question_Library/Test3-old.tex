\documentclass[12pt, addpoints]{exam}
\usepackage[left=0.1in, right=0.1in, top=0.6in,bottom=0.1in]{geometry}
\usepackage{enumitem}
\usepackage{amsmath}
\usepackage{tikz}
\usepackage{ulem}
\usetikzlibrary{shapes.geometric, arrows.meta, positioning}

% --- Tikz set for diagram
\tikzset{
    box/.style={rectangle, draw, minimum width=1.8cm, minimum height=1cm, align=left, font=\small},
    doublebox/.style={rectangle, draw, double, double distance=1.5pt, minimum width=2.5cm, minimum height=1cm, align=left, font=\small},
    decision/.style={diamond, draw, minimum width=1cm, minimum height=1cm, align=center, aspect=2, font=\small},
    doubledecision/.style={diamond, double, draw, minimum width=1cm, minimum height=1cm, align=center, aspect=2, font=\small},
    attribute/.style={rectangle, draw, minimum width=1.5cm, minimum height=0.6cm, align=center, font=\small},
    arrow/.style={-Stealth, thick},
    doublearrow/.style={-Stealth, thick, double, double distance=1.5pt},
    dashedline/.style={thick, dashed},
    line/.style={thick},
    doubleline/.style={thick, double, double distance=1.5pt}
}

% --- Make choices use open circles instead of letters ---
\renewcommand{\choicelabel}{\raisebox{0.15ex}{$\bigcirc$}}

\begin{document}
\noindent Name:\ \makebox[3in]{\hrulefill}\hfill Points Scored: \underline{\hspace{3em}} / \numpoints
\begin{questions}

% ------------------ TRUE/FALSE SECTION ------------------
\section*{Part I — True/False}

\begin{minipage}{\linewidth}
\question[1]
True or False: Definition of the \texttt{prereq} table could be changed from\par\vspace{0.5em}
\begin{verbatim}
create table prereq
(course_id          varchar(8),
 prereq_id          varchar(8),
 primary key (course_id, prereq_id),
 foreign key (course_id) 
   references course(course_id)
   on delete cascade,
 foreign key (prereq_id)
   references course(course_id)
);
\end{verbatim}
\vspace{0.5em}
to\par\vspace{0.5em}
\begin{verbatim}
create table prereq
(course_id          varchar(8),
 prereq_id          varchar(8),
 primary key (course_id, prereq_id),
 foreign key (course_id) 
   references course(course_id)
   on delete cascade,
 foreign key (prereq_id)
   references course(course_id)
   on delete set null
);
\end{verbatim}
\vspace{0.5em}
\begin{checkboxes}
    \choice True
    \CorrectChoice False
\end{checkboxes}
\end{minipage}\vspace{1em}

\begin{minipage}{\linewidth}
\question[2]
True or False: The following two inserts can be invoked, one after another, on a database instance with no tuples in any of the tables\par\vspace{0.5em}
\begin{verbatim}
insert into course values ('3300', 'DBMS', null, 3);
insert into section values ('3300', '001', 'Fall', 2022, null, null, 'A');
\end{verbatim}
\begin{checkboxes}
    \CorrectChoice True
    \choice False
\end{checkboxes}
\end{minipage}\vspace{1em}

\begin{minipage}{\linewidth}
\question[1]
True or False: Query that starts with \texttt{'select count(*) from'} is an example of a scalar query.\par\vspace{0.5em}
\begin{checkboxes}
    \choice True
    \CorrectChoice False
\end{checkboxes}
\end{minipage}\vspace{1em}

\begin{minipage}{\linewidth}
\question[2]
True or False: Below two queries are equivalent (return the same data on any instance of the University database)?\par\vspace{0.5em}
\begin{verbatim}
select * from section natural right outer join course

select * from section natural join course
\end{verbatim}
\begin{checkboxes}
    \choice True
    \CorrectChoice False
\end{checkboxes}
\end{minipage}\vspace{1em}

\begin{minipage}{\linewidth}
\question[1]
True or False: The view shown below is updatable, meaning you can insert new rows into the base tables through it.\par\vspace{0.5em}
\begin{verbatim}
create view instructor_info as 
select ID, name, building 
from instructor, department 
where instructor.dept_name = department.dept_name;
\end{verbatim}
\begin{checkboxes}
    \choice True
    \CorrectChoice False
\end{checkboxes}
\end{minipage}\vspace{1em}

\begin{minipage}{\linewidth}
\question[1]
True or False: When deleting a tuple from the \textbf{advisor} relation, the tuple itself is not removed; instead, its \textbf{i\_id} attribute is set to NULL.\par\vspace{0.5em}
\begin{checkboxes}
    \choice True
    \CorrectChoice False
\end{checkboxes}
\end{minipage}\vspace{1em}

\begin{minipage}{\linewidth}
\question[2]
True or False: Materialized views are temporary views that exist only for the duration of the current database session.\par\vspace{0.5em}
\begin{checkboxes}
    \choice True
    \CorrectChoice False
\end{checkboxes}
\end{minipage}\vspace{1em}

\begin{minipage}{\linewidth}
\question[2]
True or False: A single command is, by default, executed in one transaction, ensuring that changes are either committed entirely or rolled back entirely.\par\vspace{0.5em}
\begin{checkboxes}
    \CorrectChoice True
    \choice False
\end{checkboxes}
\end{minipage}\vspace{1em}

% ------------------ MULTIPLE CHOICE SECTION ------------------
\section*{Part II — Multiple choice (choose one)}

\begin{minipage}{\linewidth}
\question[2]
What is the maximal number of tables from which tuple(s) can be automatically deleted with the deletion of a course?\par\vspace{0.5em}
\begin{checkboxes}
    \CorrectChoice 4
    \choice 8
    \choice 1
    \choice 3
\end{checkboxes}
\end{minipage}\vspace{1em}

\begin{minipage}{\linewidth}
\question[2]
If there are 5 tuples in the \textbf{advisor} relation and 7 tuples in the \textbf{instructor} relation, 
then what is the minimal number of tuples the below query can return?\par\vspace{0.5em}

\texttt{select * from advisor full outer join instructor on (id = i\_id)}\par\vspace{0.5em}

\begin{checkboxes}
    \choice 5
    \CorrectChoice 7
    \choice 0
    \choice 12
\end{checkboxes}
\end{minipage}\vspace{1em}

\begin{minipage}{\linewidth}
\question[2]
What does the below query return?\par\vspace{0.5em}

\texttt{select takes.ID, teaches.ID from takes join teaches using (course\_id, sec\_id, semester, year)}\par
\texttt{group by takes.ID, teaches.ID having count(teaches.ID)>2}\par\vspace{0.5em}

\begin{checkboxes}
    \CorrectChoice IDs of students along with IDs of instructors who taught them more than 2 course sections.
    \choice IDs of students along with IDs of their instructors who taught more than 2 course sections
    \choice IDs of students along with IDs of their instructors who taught more than 2 sections of a course they shared
    \choice IDs of students along with IDs of instructors who taught more than 2 course sections.
\end{checkboxes}
\end{minipage}\vspace{1em}

\begin{minipage}{\linewidth}
\question[4]
Which query returns sorted data from the \textit{time\_slot} relation by \texttt{time\_slot\_id} and then day.\par\vspace{0.5em}

\begin{checkboxes}
    \choice select * from time\_slot order by time\_slot\_id, day;
    \CorrectChoice with temp(day, no) as ((select 'W', 3) union (select 'H', 4) union (select 'F', 5) union
(select 'M', 1) union (select 'T', 2)) select * from time\_slot natural join temp order by time\_slot\_id, no;
    \choice with temp(day, no) as ((select 'W', 3) union (select 'H', 4) union
(select 'F', 5) union (select 'M', 1) union (select 'T', 2))
select * from time\_slot natural join temp order by no, time\_slot\_id;
    \choice select * from time\_slot order by day, time\_slot\_id;
\end{checkboxes}
\end{minipage}\vspace{1em}

% ------------------ MULTIPLE SELECT SECTION ------------------
\section*{Part III — Multiple select (select all that apply)}

\begin{minipage}{\linewidth}
\question[4]
Choose tables from which tuples can not be deleted via deletes from other tables.\par\vspace{0.5em}
(Select all that apply)\par\vspace{0.5em}
\begin{checkboxes}
    \choice section
    \CorrectChoice classroom
    \choice takes
    \choice teaches
    \CorrectChoice time\_slot
    \CorrectChoice course
    \choice prereq
    \CorrectChoice student
    \CorrectChoice instructor
    \CorrectChoice department
    \choice advisor
\end{checkboxes}
%---\begin{solution}
%--- format for giving solution descriptions
%---\end{solution}
\end{minipage}\vspace{1em}

\begin{minipage}{\linewidth}
\question[4]
Assume that all the below commands were run on a University database and server with users: Grey, Amit, Green and Brown. 
Which users, if any, have permission to find out for each student of the 'Comp. Sci.' 
Department the number of course sections he failed (received a grade 'F' for).\par\vspace{0.5em}

\texttt{create role teaching\_assistant;}\par
\texttt{create role lecturer;}\par
\texttt{create role dean;}\par\vspace{0.5em}
\texttt{grant select on takes to teaching\_assistant;}\par
\texttt{grant update on takes to teaching\_assistant;}\par\vspace{0.5em}
\texttt{grant select on student to lecturer;}\par
\texttt{grant select on instructor to dean;}\par
\texttt{grant teaching\_assistant to lecturer;}\par
\texttt{grant lecturer to dean;}\par\vspace{0.5em}
\texttt{grant dean to Grey;}\par
\texttt{grant dean to Amit;}\par
\texttt{grant update on instructor to Amit;}\par\vspace{0.5em}
\texttt{grant lecturer to Brown;}\par
\texttt{grant update on student to Brown;}\par\vspace{0.5em}
\texttt{grant teaching\_assistant to Green;}\par

(Select all that apply)\par\vspace{0.5em}
\begin{checkboxes}
    \CorrectChoice Amit
    \CorrectChoice Brown
    \choice Green
    \CorrectChoice Grey
\end{checkboxes}
%---\begin{solution}
%--- format for giving solution descriptions
%---\end{solution}
\end{minipage}\vspace{1em}

\begin{minipage}{\linewidth}
\question[2]
Which of the given statements are true of a view?\par\vspace{0.5em}

(Select all that apply)\par\vspace{0.5em}
\begin{checkboxes}
    \CorrectChoice Some reasons for creating views are security and convenience.
    \CorrectChoice Whenever we evaluate a query in which a view relation occurs, that view relation is replaced with the saved query expression that defines that view.
    \choice A view is not reusable.
    \choice Whenever we define a view, the database system stores the result of evaluation of the query expression that defines the view, rather than the definition of the view.
\end{checkboxes}
%---\begin{solution}
%--- format for giving solution descriptions
%---\end{solution}
\end{minipage}\vspace{1em}

\begin{minipage}{\linewidth}
\question[2]
Choose constraints on a single relation.\par\vspace{0.5em}

(Select all that apply)\par\vspace{0.5em}
\begin{checkboxes}
    \CorrectChoice not null
    \CorrectChoice primary key
    \choice foreign key
    \CorrectChoice unique
\end{checkboxes}
%---\begin{solution}
%--- format for giving solution descriptions
%---\end{solution}
\end{minipage}\vspace{1em}

\begin{minipage}{\linewidth}
\question[2]
Which query/queries will use some index?\par\vspace{0.5em}

(Select all that apply)\par\vspace{0.5em}
\begin{checkboxes}
    \CorrectChoice select * from instructor where ID = '10101'
    \choice select * from instructor where ID = 10101
    \choice select * from instructor where ID = 0001
    \choice select * from instructor where name = 'Wu'
\end{checkboxes}
%---\begin{solution}
%--- format for giving solution descriptions
%---\end{solution}
\end{minipage}\vspace{1em}

\begin{minipage}{\linewidth}
\question[4]
Considering the provided (below) commands executed on a University database and server with users Grey, Amit, Green, and Brown, 
which user(s) have the authority to enter student's \textbf{grade}, for some course section, and also update his \textbf{tot\_cred}.\par\vspace{0.5em}

\texttt{create role 'teaching\_assistant'@'localhost';}\par
\texttt{create role 'lecturer'@'localhost';}\par
\texttt{create role 'dean'@'localhost';}\par\vspace{0.5em}
\texttt{grant select on university.takes to 'teaching\_assistant'@'localhost';}\par
\texttt{grant update on university.takes to 'teaching\_assistant'@'localhost';}\par\vspace{0.5em}
\texttt{grant select on university.student to 'lecturer'@'localhost';}\par
\texttt{grant select on university.instructor to 'dean'@'localhost';}\par
\texttt{grant 'teaching\_assistant'@'localhost' to 'lecturer'@'localhost';}\par
\texttt{grant 'lecturer'@'localhost' to 'dean'@'localhost';}\par\vspace{0.5em}
\texttt{grant 'dean'@'localhost' to 'Grey'@'localhost';}\par
\texttt{grant 'dean'@'localhost' to 'Amit'@'localhost';}\par
\texttt{grant update on university.instructor to 'Amit'@'localhost';}\par\vspace{0.5em}
\texttt{grant 'lecturer'@'localhost' to 'Brown'@'localhost';}\par
\texttt{grant update on university.student to 'Brown'@'localhost';}\par
\texttt{grant 'teaching\_assistant'@'localhost' to 'Green'@'localhost';}\par\vspace{0.5em}

\textbf{Hint: } To test it in MYSQL you need to invoke the following commands.\par\vspace{0.5em}
SET DEFAULT ROLE ALL TO 'Brown'@'localhost';\par
SET DEFAULT ROLE ALL TO 'Grey'@'localhost';\par
SET DEFAULT ROLE ALL TO 'Amit'@'localhost';\par
SET DEFAULT ROLE ALL TO 'Green'@'localhost'\par
(More on that in the MYSQL documentation under MySQL :: MySQL 8.0 Reference Manual :: 8.2.10 Using Roles .)\par\vspace{0.5em}
You \textbf{must} do the following \textbf{after logging in as each user} (Amit, Grey, Brown, or Green):\par\vspace{0.5em}
\textbf{Activate all roles granted to the user}: SET ROLE ALL;\par
\textbf{Select the university database}: USE university;\par
(Select all that apply)\par\vspace{0.5em}
\begin{checkboxes}
    \choice Amit
    \CorrectChoice Brown
    \choice Green
    \choice Grey
\end{checkboxes}
%---\begin{solution}
%--- format for giving solution descriptions
%---\end{solution}
\end{minipage}\vspace{1em}

% ------------------ Fill In The Blank SECTION ------------------
\section*{Part IV — Fill In The Blank}

\begin{minipage}{\linewidth}
\question[4]
A \fillin[][8em] \ifprintanswers\textbf{(transaction)}\fi is treated as a logical unit of work on the database and can have one of two outcomes. 
If it completes successfully, it is said to have \fillin[][8em] \ifprintanswers\textbf{(committed)}\fi. 
If it does not execute successfully, it is \fillin[][8em] \ifprintanswers\textbf{(aborted, rolled back, undone)}\fi.
\end{minipage}\vspace{1em}
\end{questions}
\end{document}