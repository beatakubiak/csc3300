\documentclass[12pt, addpoints, answers]{exam}
\usepackage[left=0.1in, right=0.1in, top=0.6in,bottom=0.1in]{geometry}
\usepackage{enumitem}
\usepackage{amsmath}
\usepackage{tikz}
\usepackage{ulem}
\usetikzlibrary{shapes.geometric, arrows.meta, positioning}

% --- Tikz set for diagram
\tikzset{
    box/.style={rectangle, draw, minimum width=1.8cm, minimum height=1cm, align=left, font=\small},
    doublebox/.style={rectangle, draw, double, double distance=1.5pt, minimum width=2.5cm, minimum height=1cm, align=left, font=\small},
    decision/.style={diamond, draw, minimum width=1cm, minimum height=1cm, align=center, aspect=2, font=\small},
    doubledecision/.style={diamond, double, draw, minimum width=1cm, minimum height=1cm, align=center, aspect=2, font=\small},
    attribute/.style={rectangle, draw, minimum width=1.5cm, minimum height=0.6cm, align=center, font=\small},
    arrow/.style={-Stealth, thick},
    doublearrow/.style={-Stealth, thick, double, double distance=1.5pt},
    dashedline/.style={thick, dashed},
    line/.style={thick},
    doubleline/.style={thick, double, double distance=1.5pt}
}

% --- Make choices use open circles instead of letters ---
\renewcommand{\choicelabel}{\raisebox{0.15ex}{$\bigcirc$}}

% Header/footer formatting
\pagestyle{headandfoot}
\runningheadrule
\firstpageheadrule
\firstpageheader{Database Management Systems}{Ch 8 Old}{Fall 2025}
\runningheader{}{Ch 8 Old}{}
\runningfooter{}{}{}

\begin{document}
\noindent Name:\ \makebox[3in]{\hrulefill}\hfill Points Scored: \underline{\hspace{3em}} / \numpoints
\begin{questions}

% ------------------ TRUE/FALSE SECTION ------------------
\section*{Part I — True/False}

\begin{minipage}{\linewidth}
\question[2]
True or False: In the result of the provided query, there might be instructors whose information (ID, name, dept\_name, salary) is not included.\par\vspace{0.5em}
\begin{verbatim}
select * from instructor natural left outer join teaches
\end{verbatim}
\begin{checkboxes}
    \choice True
    \CorrectChoice False
\end{checkboxes}
\end{minipage}\vspace{1em}

\begin{minipage}{\linewidth}
\question[2]
True or False: Below two queries are equivalent (return the same data on any instance of the University database)?\par\vspace{0.5em}
\begin{verbatim}
select * from section natural right outer join course

select * from section natural join course
\end{verbatim}
\begin{checkboxes}
    \choice True
    \CorrectChoice False
\end{checkboxes}
\end{minipage}\vspace{1em}

\begin{minipage}{\linewidth}
\question[2]
True or False: Deletion of a tuple from the \textbf{department} relation may result in the deletion of some tuples from other relations.\par\vspace{0.5em}
\begin{checkboxes}
    \choice True
    \CorrectChoice False
\end{checkboxes}
\end{minipage}\vspace{1em}

\begin{minipage}{\linewidth}
\question[2]
True or False: Once a transaction has executed commit work, its effects can no longer be undone by rollback work.\par\vspace{0.5em}
\begin{checkboxes}
    \CorrectChoice True
    \choice False
\end{checkboxes}
\end{minipage}\vspace{1em}

\begin{minipage}{\linewidth}
\question[2]
True or False: Consider a view created with the following command.\par\vspace{0.5em}
\begin{verbatim}
create view faculty as
select id, name, dept_name
from instructor
\end{verbatim}
The following two queries return the same data for any instance of the University database.\par\vspace{0.5em}
\begin{verbatim}
select count(*) from instructor
select count(*) from faculty
\end{verbatim}
\begin{checkboxes}
    \choice True
    \CorrectChoice False
\end{checkboxes}
\end{minipage}\vspace{1em}

\begin{minipage}{\linewidth}
\question[2]
True or False: Definition of the \textit{prereq} table could be changed from: \par\vspace{0.5em}
\begin{verbatim}
create table prereq
(course_id        varchar(8), 
prereq_id        varchar(8),
primary key (course_id, prereq_id),
foreign key (course_id) references course(course_id) on delete cascade,
foreign key (prereq_id) references course(course_id)
);
\end{verbatim}
to\par\vspace{0.5em}
\begin{verbatim}
create table prereq
(course_id        varchar(8), 
prereq_id        varchar(8),
primary key (course_id, prereq_id),
foreign key (course_id) references course(course_id) on delete cascade,
foreign key (prereq_id) references course(course_id) on delete cascade
);
\end{verbatim}
\begin{checkboxes}
    \CorrectChoice True
    \choice False
\end{checkboxes}
\end{minipage}\vspace{1em}

\begin{minipage}{\linewidth}
\question[2]
True or False: Deleting a tuple from the \textbf{instructor} relation can result in setting value(s) of some attribute(s) 
of some tuples in the \textbf{advisor} relation to NULL.\par\vspace{0.5em}
\begin{checkboxes}
    \CorrectChoice True
    \choice False
\end{checkboxes}
\end{minipage}\vspace{1em}

\begin{minipage}{\linewidth}
\question[2]
True or False: If we try to delete a tuple from the \textbf{advisor} relation, the tuple is not deleted. 
Instead, \textbf{i\_id} of this tuple is set to NULL.\par\vspace{0.5em}
\begin{checkboxes}
    \choice True
    \CorrectChoice False
\end{checkboxes}
\end{minipage}\vspace{1em}

\begin{minipage}{\linewidth}
\question[3]
True or False: Consider the following scenario involving a view definition and an insertion operation. 
The count of rows returned by the last command remains unchanged before and after the insertion. Is it true or false?\par\vspace{0.5em}
\begin{verbatim}
create view history_instructors as
select * 
from instructor
where dept_name= 'History';

insert into history_instructors values ('25566', 'Brown', 'Biology', 100000);

select count(*) from history_instructors;
\end{verbatim}
\begin{checkboxes}
    \CorrectChoice True
    \choice False
\end{checkboxes}
\end{minipage}\vspace{1em}

% ------------------ MULTIPLE CHOICE SECTION ------------------

\section*{Part II — Multiple choice (choose one)}

\begin{minipage}{\linewidth}
\question[2]
If there are 5 tuples in the \textbf{advisor} relation and 7 tuples in the \textbf{instructor} relation, 
then what is the minimal number of tuples the below query can return?\par\vspace{0.5em}
\begin{verbatim} 
  select * from advisor full outer join instructor on (id = i_id)
\end{verbatim}
\begin{checkboxes}
    \choice 5
    \CorrectChoice 7
    \choice 8
    \choice 12
\end{checkboxes}
\end{minipage}\vspace{1em}

\begin{minipage}{\linewidth}
\question[3]
What is the length of the longest chain of cascading deletes in the Univesrity database?\par\vspace{0.5em}
If deletion of a tuple from relation A can result in deletion of some tuple(s) from relation B, 
but deletion of a tuple from B can't result in deletion of tuple(s) from other relations, then the chain of cascading deletes, 
that starts at A, is of length 1.\par\vspace{0.5em}
\begin{checkboxes}
    \choice 1
    \CorrectChoice 2
    \choice 3
    \choice 4
\end{checkboxes}
\end{minipage}\vspace{1em}

% ------------------ MULTIPLE SELECT SECTION ------------------
\section*{Part III — Multiple select (select all that apply)}

\begin{minipage}{\linewidth}
\question[4]
Which of the given statements are true of a view?\par\vspace{0.5em}
\begin{checkboxes}
    \CorrectChoice Some reasons for creating views are security and convenience.
    \CorrectChoice Whenever we evaluate a query in which a view relation occurs, that view relation is replaced 
    with the saved query expression that defines that view.
    \choice A view is not reusable.
    \choice Whenever we define a view, the database system stores the result of evaluation of the query 
    expression that defines the view, rather than the definition of the view.
\end{checkboxes}
\end{minipage}\vspace{1em}

\end{questions}
\end{document}