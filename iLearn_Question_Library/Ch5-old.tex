\documentclass[12pt, addpoints, answers]{exam}
\usepackage[left=0.1in, right=0.1in, top=0.6in,bottom=0.1in]{geometry}
\usepackage{enumitem}
\usepackage{amsmath}
\usepackage{tikz}
\usepackage{ulem}
\usetikzlibrary{shapes.geometric, arrows.meta, positioning}

% --- Tikz set for diagram
\tikzset{
    box/.style={rectangle, draw, minimum width=1.8cm, minimum height=1cm, align=left, font=\small},
    doublebox/.style={rectangle, draw, double, double distance=1.5pt, minimum width=2.5cm, minimum height=1cm, align=left, font=\small},
    decision/.style={diamond, draw, minimum width=1cm, minimum height=1cm, align=center, aspect=2, font=\small},
    doubledecision/.style={diamond, double, draw, minimum width=1cm, minimum height=1cm, align=center, aspect=2, font=\small},
    attribute/.style={rectangle, draw, minimum width=1.5cm, minimum height=0.6cm, align=center, font=\small},
    arrow/.style={-Stealth, thick},
    doublearrow/.style={-Stealth, thick, double, double distance=1.5pt},
    dashedline/.style={thick, dashed},
    line/.style={thick},
    doubleline/.style={thick, double, double distance=1.5pt}
}

% --- Make choices use open circles instead of letters ---
\renewcommand{\choicelabel}{\raisebox{0.15ex}{$\bigcirc$}}

% Header/footer formatting
\pagestyle{headandfoot}
\runningheadrule
\firstpageheadrule
\firstpageheader{Database Management Systems}{Ch 5 Old}{Fall 2025}
\runningheader{}{Ch 5 Old}{}
\runningfooter{}{}{}

\begin{document}
\noindent Name:\ \makebox[3in]{\hrulefill}\hfill Points Scored: \underline{\hspace{3em}} / \numpoints
\begin{questions}

% ------------------ TRUE/FALSE SECTION ------------------
\section*{Part I — True/False}

\begin{minipage}{\linewidth}
\question[2]
True or False: JDBC drivers implement the JDBC API.\par\vspace{0.5em}
\begin{checkboxes}
    \CorrectChoice True
    \choice False
\end{checkboxes}
\end{minipage}\vspace{1em}

\begin{minipage}{\linewidth}
\question[2]
True or False: A function has only input parameters and can return multiple values.\par\vspace{0.5em}
\begin{checkboxes}
    \choice True
    \CorrectChoice False
\end{checkboxes}
\end{minipage}\vspace{1em}

\begin{minipage}{\linewidth}
\question[2]
True or False: In a database management system, the execution of a trigger can cause another trigger to execute.\par\vspace{0.5em}
\begin{checkboxes}
    \CorrectChoice True
    \choice False
\end{checkboxes}
\end{minipage}\vspace{1em}

\begin{minipage}{\linewidth}
\question[2]
True or False: If f was a stored function with one parameter of type varchar(20) returning an integer, then the following command would work.\par\vspace{0.5em}
select f('HelloThere') = 1;\par\vspace{0.5em}
\begin{checkboxes}
    \CorrectChoice True
    \choice False
\end{checkboxes}
\end{minipage}\vspace{1em}

\begin{minipage}{\linewidth}
\question[2]
True or False: MySQL Connector/J implements the JDBC API.\par\vspace{0.5em}
\begin{checkboxes}
    \CorrectChoice True
    \choice False
\end{checkboxes}
\end{minipage}\vspace{1em}

\begin{minipage}{\linewidth}
\question[4]
True or False: The execution of this Java code snippet inserts exactly one record into the instructor table.\par\vspace{0.5em}
\begin{verbatim}
String id = "225";
String name = "Green";
String dept_name = "Comp. Sci.";
BigDecimal salary = new BigDecimal(98000);
PreparedStatement pstmt = conn.prepareStatement
("insert into instructor values (?, ?, ?, ?)");
pstmt.setString(1, id);
pstmt.setString(2, name);
pstmt.setString(3, dept_name);
pstmt.setBigDecimal(4, salary);
pstmt.executeUpdate();
id = "226"
pstmt.setString(1, id);
pstmt.executeUpdate();
\end{verbatim}
Assume the code runs successfully: there are no instructors with the given IDs and there is a 'Comp. Sci.' department.\par\vspace{0.5em}
\begin{checkboxes}
    \choice True
    \CorrectChoice False
\end{checkboxes}
\end{minipage}\vspace{1em}

\begin{minipage}{\linewidth}
\question[2]
True or False: A trigger can only execute after a specified operation (INSERT, UPDATE, DELETE) has completed.\par\vspace{0.5em}
\begin{checkboxes}
    \choice True
    \CorrectChoice False
\end{checkboxes}
\end{minipage}\vspace{1em}

\begin{minipage}{\linewidth}
\question[2]
True or False: Materialized views physically store the result set of a query, and their data can be updated periodically.\par\vspace{0.5em}
\begin{checkboxes}
    \CorrectChoice True
    \choice False
\end{checkboxes}
\end{minipage}\vspace{1em}

\begin{minipage}{\linewidth}
\question[2]
True or False: A primary key constraint automatically creates an index.\par\vspace{0.5em}
\begin{checkboxes}
    \CorrectChoice True
    \choice False
\end{checkboxes}
\end{minipage}\vspace{1em}

\begin{minipage}{\linewidth}
\question[2]
True or False: Dynamic SQL allows the program to construct an SQL query as a character string at runtime, submit the query, 
and then retrieve the result into program variables a tuple at a time.\par\vspace{0.5em}
\begin{checkboxes}
    \CorrectChoice True
    \choice False
\end{checkboxes}
\end{minipage}\vspace{1em}

% ------------------ MULTIPLE CHOICE SECTION ------------------

\section*{Part II — Multiple choice (choose one)}

\begin{minipage}{\linewidth}
\question[2]
The tester checks the application (e.g. Web application) for security vulnerabilities. 
Which character is usually used first to check if SQL injection is possible?\par\vspace{0.5em}
\begin{checkboxes} 
    \choice \$ (dollar)
    \choice " (double quote)
    \choice ; (semicolon)
    \CorrectChoice ' (single quote)
\end{checkboxes}
\end{minipage}\vspace{1em}

\begin{minipage}{\linewidth}
\question[3]
Which code snippet is vulnerable to SQL injection attacks?\par\vspace{0.5em}
\begin{checkboxes} 
    \CorrectChoice \begin{verbatim}
String id = "225";
String name = "Green";
String dept_name = "Comp. Sci.";
BigDecimal salary = new BigDecimal(98000);
Statement stmt = conn.createStatement();
stmt.executeUpdate("insert into instructor values 
('" + id + "', '" + name + "', '" + dept_name + "', " + salary + ")");
    \end{verbatim}
    \choice \begin{verbatim}
String id = "225";
String name = "Green";
String dept_name = "Comp. Sci.";
BigDecimal salary = new BigDecimal(98000);
PreparedStatement pstmt = conn.prepareStatement
("insert into instructor values (?, ?, ?, ?)");
pstmt.setString(1, id);
pstmt.setString(2, name);
pstmt.setString(3, dept_name);
pstmt.setBigDecimal(4, salary);
pstmt.executeUpdate();
    \end{verbatim}
\end{checkboxes}
\end{minipage}\vspace{1em}

% ------------------ MULTIPLE SELECT SECTION ------------------
\section*{Part III — Multiple select (select all that apply)}

\begin{minipage}{\linewidth}
\question[6]
Which values, if any, can TODO be replaced with, so that the below query runs? (Select all that apply)\par\vspace{0.5em}
\begin{verbatim}
INSERT INTO instructor values ('6667', TODO, 'Comp. Sci.', null);
\end{verbatim}
(Select all that apply)\par\vspace{0.5em}
\begin{checkboxes}
    \CorrectChoice (SELECT VERSION())
    \CorrectChoice 'Green', null, null); -- (the two minus signs are followed by a space)
    \CorrectChoice (select count(*) from information\_schema.schemata)
    \choice '
\end{checkboxes}
\end{minipage}\vspace{1em}

\begin{minipage}{\linewidth}
\question[4]
Consider the following function definition and the query that follows?\par\vspace{0.5em}
\begin{verbatim}
create function instructors_no_fun
(dept_name varchar(20))
returns integer
begin
declare d_count integer;
select count(*) into d_count 
from instructor where instructor.dept_name = dept_name;
return d_count;
end

select dept_name, instructors_no_fun(dept_name)
from department
\end{verbatim}
Hint: If you get an error when creating the function, try running:\par\vspace{0.5em}
\begin{verbatim}
SET GLOBAL log_bin_trust_function_creators = 1;
\end{verbatim}
This allows function creation when binary logging is enabled.\par\vspace{0.5em}
(Select all that apply)\par\vspace{0.5em}
\begin{checkboxes}
    \CorrectChoice dept\_name, of type varchar(20), is the only parameter of the function instructors\_no\_fun.
    \CorrectChoice The query returns, for each department, the number of instructors associated with it.
    \choice For departments with instructors, the query returns the instructor count.
    \CorrectChoice The query runs function instructors\_no\_fun as many times as there are tuples in the table department.
\end{checkboxes}
\end{minipage}\vspace{1em}

\begin{minipage}{\linewidth}
\question[4]
Consider the following procedure definition and commands that follow?\par\vspace{0.5em}
\begin{verbatim}
create procedure instructors_no_proc
(in dept_name varchar(20), out d_count integer)
begin
select count(*) into d_count 
from instructor where instructor.dept_name = dept_name;
end

call instructors_no_proc('Comp. Sci.', @a)

select @a
\end{verbatim}
(Select all that apply)\par\vspace{0.5em}
\begin{checkboxes}
    \CorrectChoice The last command is a scalar query.
    \choice The following syntax is valid for calling the instructors\_no\_proc procedure:
    \begin{verbatim}
    select instructors_no_proc('Comp. Sci.', @a)
    \end{verbatim}
    \choice The following command calls the instructors\_no\_proc procedure correctly:
    \begin{verbatim}
    call instructors_no_proc('Comp. Sci.', 7)
    \end{verbatim}
    \choice The execution of instructors\_no\_proc procedure can result in a modification to the database data.
\end{checkboxes}
\end{minipage}\vspace{1em}

\end{questions}
\end{document}