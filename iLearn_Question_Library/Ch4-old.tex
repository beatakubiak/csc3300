\documentclass[12pt, addpoints, answers]{exam}
\usepackage[left=0.1in, right=0.1in, top=0.6in,bottom=0.1in]{geometry}
\usepackage{enumitem}
\usepackage{amsmath}
\usepackage{tikz}
\usepackage{ulem}
\usetikzlibrary{shapes.geometric, arrows.meta, positioning}

% --- Tikz set for diagram
\tikzset{
    box/.style={rectangle, draw, minimum width=1.8cm, minimum height=1cm, align=left, font=\small},
    doublebox/.style={rectangle, draw, double, double distance=1.5pt, minimum width=2.5cm, minimum height=1cm, align=left, font=\small},
    decision/.style={diamond, draw, minimum width=1cm, minimum height=1cm, align=center, aspect=2, font=\small},
    doubledecision/.style={diamond, double, draw, minimum width=1cm, minimum height=1cm, align=center, aspect=2, font=\small},
    attribute/.style={rectangle, draw, minimum width=1.5cm, minimum height=0.6cm, align=center, font=\small},
    arrow/.style={-Stealth, thick},
    doublearrow/.style={-Stealth, thick, double, double distance=1.5pt},
    dashedline/.style={thick, dashed},
    line/.style={thick},
    doubleline/.style={thick, double, double distance=1.5pt}
}

% --- Make choices use open circles instead of letters ---
\renewcommand{\choicelabel}{\raisebox{0.15ex}{$\bigcirc$}}

% Header/footer formatting
\pagestyle{headandfoot}
\runningheadrule
\firstpageheadrule
\firstpageheader{Database Management Systems}{Ch4 Old}{Fall 2025}
\runningheader{}{Ch4 Old}{}
\runningfooter{}{}{}

\begin{document}
\noindent Name:\ \makebox[3in]{\hrulefill}\hfill Points Scored: \underline{\hspace{3em}} / \numpoints
\begin{questions}

% ------------------ TRUE/FALSE SECTION ------------------
\section*{Part I — True/False}

\begin{minipage}{\linewidth}
\question[2]
True or False: The DBMS will utilize an index to retrieve the result of the following query:\par
\begin{verbatim}
select * 
from course
where course_id = 3300
\end{verbatim}
\begin{checkboxes}
    \choice True
    \CorrectChoice False
\end{checkboxes}
\end{minipage}\vspace{1em}

\begin{minipage}{\linewidth}
\question[2]
True or False: Deletion of a tuple from the \textbf{department} relation may result in the deletion of some tuples from other relations.\par\vspace{0.5em}
\begin{checkboxes}
    \choice True
    \CorrectChoice False
\end{checkboxes}
\end{minipage}\vspace{1em}

\begin{minipage}{\linewidth}
\question[2]
True or False: Indices are data structures used to speed up access to records with specified values for index attributes.\par\vspace{0.5em}
\begin{checkboxes}
    \CorrectChoice True
    \choice False
\end{checkboxes}
\end{minipage}\vspace{1em}

\begin{minipage}{\linewidth}
\question[2]
True or False: It is possible to delete successfully any tuple from any relation of any University Database instance.\par\vspace{0.5em}
\begin{checkboxes}
    \choice True
    \CorrectChoice False
\end{checkboxes}
\end{minipage}\vspace{1em}

\begin{minipage}{\linewidth}
\question[2]
True or False: If we try to delete a tuple from the \textbf{advisor} relation, the tuple is not deleted. Instead, \textbf{i\_id} of this tuple is set to NULL.\par\vspace{0.5em}
\begin{checkboxes}
    \choice True
    \CorrectChoice False
\end{checkboxes}
\end{minipage}\vspace{1em}

\begin{minipage}{\linewidth}
\question[2]
True or False: Deletion of a tuple from the \textbf{department} relation may result in the deletion of some tuple(s) from other relation(s).\par\vspace{0.5em}
\begin{checkboxes}
    \choice True
    \CorrectChoice False
\end{checkboxes}
\end{minipage}\vspace{1em}

\begin{minipage}{\linewidth}
\question[2]
True or False: Deleting a tuple from the \textbf{instructor} relation can result in setting value(s) of some attribute(s) 
of some tuples in the \textbf{advisor} relation to NULL.\par\vspace{0.5em}
\begin{checkboxes}
    \CorrectChoice True
    \choice False
\end{checkboxes}
\end{minipage}\vspace{1em}

\begin{minipage}{\linewidth}
\question[3]
True or False: The first list of statements is equivalent to the last statement 
(they update the database in the same way and each of them is executed atomically).\par
\begin{verbatim}
start transaction;
update department set budget = budget - 500 where dept_name = 'Biology';
update department set budget = budget + 500 where dept_name = 'Comp. Sci.';
commit;

update department withdraw, department deposit
set withdraw.budget = withdraw.budget - 500,
    deposit.budget = deposit.budget + 500
where withdraw.dept_name = 'Biology' and deposit.dept_name = 'Comp. Sci.';
\end{verbatim}
\begin{checkboxes}
    \CorrectChoice True
    \choice False
\end{checkboxes}
\end{minipage}\vspace{1em}

\begin{minipage}{\linewidth}
\question[2]
True or False: Consider relation R with attributes A, B, C, D. Constraint \textit{unique} (A, B) on this relation 
ensures that no two of its tuples have the same values for attributes A and B.\par\vspace{0.5em}
\begin{checkboxes}
    \CorrectChoice True
    \choice False
\end{checkboxes}
\end{minipage}\vspace{1em}

\begin{minipage}{\linewidth}
\question[2]
True or False: Once a transaction has executed commit work, its effects can no longer be undone by rollback work.\par\vspace{0.5em}
\begin{checkboxes}
    \CorrectChoice True
    \choice False
\end{checkboxes}
\end{minipage}\vspace{1em}

\begin{minipage}{\linewidth}
\question[2]
True or False: View names can be used in a query wherever a relation name is allowed.\par\vspace{0.5em}
\begin{checkboxes}
    \CorrectChoice True
    \choice False
\end{checkboxes}
\end{minipage}\vspace{1em}

\begin{minipage}{\linewidth}
\question[2]
True or False: Consider a view created with the following command.\par\vspace{0.5em}
\begin{verbatim}
create view faculty as
select id, name, dept_name
from instructor
\end{verbatim}
The following two queries return the same data for any instance of the University database.\par\vspace{0.5em}
\begin{verbatim}
select count(*) from instructor
select count(*) from faculty
\end{verbatim}
\begin{checkboxes}
    \CorrectChoice True
    \choice False
\end{checkboxes}
\end{minipage}\vspace{1em}

\begin{minipage}{\linewidth}
\question[2]
True or False: Definition of the \textit{prereq} table could be changed from: \par\vspace{0.5em}
\begin{verbatim}
create table prereq
(course_id        varchar(8), 
prereq_id        varchar(8),
primary key (course_id, prereq_id),
foreign key (course_id) references course(course_id) on delete cascade,
foreign key (prereq_id) references course(course_id)
);
\end{verbatim}
to\par\vspace{0.5em}
\begin{verbatim}
create table prereq
(course_id        varchar(8), 
prereq_id        varchar(8),
primary key (course_id, prereq_id),
foreign key (course_id) references course(course_id) on delete cascade,
foreign key (prereq_id) references course(course_id) on delete cascade
);
\end{verbatim}
\begin{checkboxes}
    \CorrectChoice True
    \choice False
\end{checkboxes}
\end{minipage}\vspace{1em}

\begin{minipage}{\linewidth}
\question[3]
True or False: Consider the following scenario involving a view definition and an insertion operation. 
The count of rows returned by the last command remains unchanged before and after the insertion. Is it true or false?\par\vspace{0.5em}
\begin{verbatim}
create view history_instructors as
select * 
from instructor
where dept_name= 'History';

insert into history_instructors values ('25566', 'Brown', 'Biology', 100000);

select count(*) from history_instructors;
\end{verbatim}
\begin{checkboxes}
    \CorrectChoice True
    \choice False
\end{checkboxes}
\end{minipage}\vspace{1em}

\begin{minipage}{\linewidth}
\question[2]
True or False: Deleting a tuple from the \textbf{student} relation can result in setting value(s) of some attribute(s) 
of some tuples in the \textbf{advisor} relation to NULL.\par\vspace{0.5em}
\begin{checkboxes}
    \choice True
    \CorrectChoice False
\end{checkboxes}
\end{minipage}\vspace{1em}

\begin{minipage}{\linewidth}
\question[2]
True or False: The DBMS will utilize an index to retrieve the result of the following query:\par\vspace{0.5em}
\begin{verbatim}
select * 
from student 
where id = 12345
\end{verbatim}
HINT: MYSQL: explain\par\vspace{0.5em}
\begin{checkboxes}
    \choice True
    \CorrectChoice False
\end{checkboxes}
\end{minipage}\vspace{1em}

\begin{minipage}{\linewidth}
\question[2]
True or False: A user of the database can have only one role assigned to them.\par\vspace{0.5em}
\begin{checkboxes}
    \choice True
    \CorrectChoice False
\end{checkboxes}
\end{minipage}\vspace{1em}

\begin{minipage}{\linewidth}
\question[2]
True or False: An index can be created on more than one attribute (column) of a table.\par\vspace{0.5em}
HINT: Below is a command that creates and index on the name attribute of the table student.\par\vspace{0.5em}
\begin{verbatim}
create index studentName_index on student(name)
\end{verbatim}
\begin{checkboxes}
    \CorrectChoice True
    \choice False
\end{checkboxes}
\end{minipage}\vspace{1em}

% ------------------ MULTIPLE CHOICE SECTION ------------------

\section*{Part II — Multiple choice (choose one)}

\begin{minipage}{\linewidth}
\question[3]
What is the effect of addition of:\par
\begin{verbatim}
unique (s_ID, i_ID)
\end{verbatim}
to the definition of the \textbf{advisor} relation?\par\vspace{0.5em}
\begin{checkboxes}
    \CorrectChoice None.
    \choice Every student is able to have at most one advisor and every instructor is able to have at most one advisee
    \choice \textbf{i\_ID} attribute can't have value null.
    \choice \textbf{s\_ID} and \textbf{i\_ID} can not both be null.
\end{checkboxes}
\end{minipage}\vspace{1em}

\begin{minipage}{\linewidth}
\question[2]
The database provides an abstraction of a transaction as being \underline{\hspace{4em}}, that is indivisible.\par\vspace{0.5em}
\begin{checkboxes}
    \choice committed
    \CorrectChoice atomic
    \choice efficient
    \choice consistent
\end{checkboxes}
\end{minipage}\vspace{1em}

\begin{minipage}{\linewidth}
\question[2]
The \underline{\hspace{4em}} clause provides a way of defining a temporary view whose definition is available only to the query in which the clause occurs.\par\vspace{0.5em}
\begin{checkboxes}
    \choice using
    \choice view
    \choice as
    \CorrectChoice with
\end{checkboxes}
\end{minipage}\vspace{1em}

\begin{minipage}{\linewidth}
\question[3]
What is the length of the longest chain of cascading deletes in the Univesrity database?\par\vspace{0.5em}
If deletion of a tuple from relation A can result in deletion of some tuple(s) from relation B, 
but deletion of a tuple from B can't result in deletion of tuple(s) from other relations, then the chain of cascading deletes, 
that starts at A, is of length 1.\par\vspace{0.5em}
\begin{checkboxes}
    \choice 1
    \CorrectChoice 2
    \choice 3
    \choice 4
\end{checkboxes}
\end{minipage}\vspace{1em}

\begin{minipage}{\linewidth}
\question[3]
Excluding constraints related to attribute domains, how many additional constraints are imposed on the attributes of 
the "department" relation in the University database?\par\vspace{0.5em}
\begin{checkboxes}
    \CorrectChoice 2
    \choice 1
    \choice 0
    \choice 4
\end{checkboxes}
\end{minipage}\vspace{1em}

\begin{minipage}{\linewidth}
\question[3]
What is the maximal number of tables from which tuple(s) can be automatically deleted with the deletion of a \textbf{student}?\par\vspace{0.5em}
\begin{checkboxes}
    \choice 1
    \choice 0
    \CorrectChoice 2
    \choice 4
\end{checkboxes}
\end{minipage}\vspace{1em}

% ------------------ MULTIPLE SELECT SECTION ------------------
\section*{Part III — Multiple select (select all that apply)}

\begin{minipage}{\linewidth}
\question[5]
Consider the following view, named myView, defined in the University database.\par\vspace{0.5em}
\begin{verbatim}
create view myView(dept_name, course_id, amount) as
       select dept_name, course_id, count(*)
       from course natural join section
       group by dept_name, course_id;
\end{verbatim}
Which of the given SQL statements are allowed on the view?\par\vspace{0.5em}
\begin{checkboxes}
    \CorrectChoice select dept\_name, course\_id from myView
    \CorrectChoice select * from course where course\_id in (select course\_id from myView)
    \CorrectChoice select course\_id, amount from myView where amount \texttt{=} (select min(amount) from myView)
    \choice create view nextView(course\_id) as select course\_id, title from myView where amount \texttt{>=} 2
    \choice insert into myView values ('Physics', 'PHY-101', 1)
\end{checkboxes}
\end{minipage}\vspace{1em}

\begin{samepage}
\question[8]
Considering the commands shown below were executed on a University database server with users Grey, Amit, Green, and Brown,
which user(s) have the authority to move all Computer Science Department students who enrolled in CS-190 during the Spring 2009 
semester to section 2 of the course (assume section 2 has already been created)?\par\vspace{0.5em}
In other words, which user(s) can successfully execute the following query?\par
(Select all that apply)\par\vspace{0.5em}
\begin{verbatim}
update takes
set sec_id = 2
where id in (select id from student where dept_name = 'Comp. Sci.')
and course_id = 'CS-190'
and semester ='Spring'
and year = 2009
\end{verbatim}
Executed Commands:\par\vspace{0.5em}
\begin{verbatim}
create role 'teaching_assistant'@'localhost';
create role 'lecturer'@'localhost';
create role 'dean'@'localhost';

grant select on university.takes to 'teaching_assistant'@'localhost';
grant update on university.takes to 'teaching_assistant'@'localhost';

grant select on university.student to 'lecturer'@'localhost';

grant select on university.instructor to 'dean'@'localhost';

grant 'teaching_assistant'@'localhost' to 'lecturer'@'localhost';
grant 'lecturer'@'localhost' to 'dean'@'localhost';

grant 'dean'@'localhost' to 'Grey'@'localhost';
grant 'dean'@'localhost' to 'Amit'@'localhost';
grant update on university.instructor to 'Amit'@'localhost';

grant 'lecturer'@'localhost' to 'Brown'@'localhost';
grant update on university.student to 'Brown'@'localhost';

grant 'teaching_assistant'@'localhost' to 'Green'@'localhost';
\end{verbatim}

Hint: \\
To test this scenario in MySQL, make sure to enable the default roles for each user by executing:\par\vspace{0.5em}
\begin{verbatim}
SET DEFAULT ROLE ALL TO  'Brown'@'localhost';
SET DEFAULT ROLE ALL TO  'Grey'@'localhost';
SET DEFAULT ROLE ALL TO  'Amit'@'localhost';
SET DEFAULT ROLE ALL TO  'Green'@'localhost';
\end{verbatim}
(More on that in the MYSQL documentation under MySQL :: MySQL 8.0 Reference Manual :: 8.2.10 Using Roles .)
Important: To correctly test this scenario, each user must run the following two commands after logging in, before executing the UPDATE query:\par\vspace{0.5em}
\begin{verbatim}
1. Activate all granted roles for the current session:
SET ROLE ALL;

2. Select the appropriate database:
USE university;
\end{verbatim}
\begin{checkboxes}
    \CorrectChoice Amit
    \CorrectChoice Brown
    \choice Green
    \CorrectChoice Grey
\end{checkboxes}
\vspace{1em}
\end{samepage}

\begin{minipage}{\linewidth}
\question[6]
Consider the following view, named myView, defined in the University database.\par\vspace{0.5em}
\begin{verbatim}
create view myView as
select name, dept_name
from instructor;
\end{verbatim}
Which of the given SQL statements are allowed on the view?\par\vspace{0.5em}
\begin{checkboxes}
    \CorrectChoice select name, dept\_name from myView
    \CorrectChoice select * from department natural join myView
    \CorrectChoice create view nextView as select dept\_name, count(*) from myView group by dept\_name
    \choice insert into myView values ('Grey', 'Comp. Sci.')
\end{checkboxes}
\end{minipage}\vspace{1em}

\begin{samepage}
\question[6]
Which statements, if any, are true about the below trigger?\par\vspace{0.5em}
\begin{verbatim}
create trigger credits_earned 
after update on takes
for each row
begin
    if (new.grade <> 'F' and new.grade is not null and 
        (old.grade = 'F' or old.grade is null)) 
    then
        update student
        set tot_cred = tot_cred + (select credits
                                   from course
                                   where course.course_id = new.course_id)
        where student.id = new.id;
    end if;
end
\end{verbatim}

\begin{checkboxes}
\CorrectChoice The following query will result in as many updates of some \textit{total\_credit} values of \textbf{student} 
as there are nulls for the \textit{grade} attribute of \textbf{takes}.
\textit{update takes set grade = 'D' where grade = null}

\choice The trigger body will be executed at least once when the query \textit{delete from takes where grade is null} 
is executed, provided some students have ungraded course sections.

\choice Assume that there are only the following tuples in the database.

\textbf{student}

\begin{tabular}{|l|l|l|l|}
\hline
id & name & dept\_name & tot\_cred \\
\hline
'5' & 'Brown' & null & 0 \\
\hline
\end{tabular}

\textbf{course}

\begin{tabular}{|l|l|l|l|}
\hline
course\_id & title & dept\_name & credits \\
\hline
'3300' & 'DBMS' & null & 3 \\
\hline
\end{tabular}

\textbf{section}

\begin{tabular}{|l|l|l|l|l|l|l|}
\hline
course\_id & sec\_id & semester & year & building & room\_number & time\_slot\_id \\
\hline
'3300' & '1' & 'Fall' & 2024 & null & null & null \\
\hline
\end{tabular}

\textbf{takes}

\begin{tabular}{|l|l|l|l|l|l|}
\hline
id & course\_id & sec\_id & semester & year & grade \\
\hline
'5' & '3300' & '1' & 'Fall' & 2024 & null \\
\hline
\end{tabular}
\begin{verbatim}
update student set tot_cred = tot_cred + 
(select sum(credits) from course natural 
join takes where takes.id = student.id and 
student.id = '5' and grade = null)
\end{verbatim}
\choice Assume that there are only the following tuples in the database.

\textbf{student}

\begin{tabular}{|l|l|l|l|}
\hline
id & name & dept\_name & tot\_cred \\
\hline
'5' & 'Brown' & null & 0 \\
\hline
\end{tabular}

\textbf{course}

\begin{tabular}{|l|l|l|l|}
\hline
course\_id & title & dept\_name & credits \\
\hline
'3300' & 'DBMS' & null & 3 \\
\hline
\end{tabular}

\textbf{section}

\begin{tabular}{|l|l|l|l|l|l|l|}
\hline
course\_id & sec\_id & semester & year & building & room\_number & time\_slot\_id \\
\hline
'3300' & '1' & 'Fall' & 2024 & null & null & null \\
\hline
\end{tabular}

\textbf{takes}

\begin{tabular}{|l|l|l|l|l|l|}
\hline
id & course\_id & sec\_id & semester & year & grade \\
\hline
'5' & '3300' & '1' & 'Fall' & 2024 & '' \\
\hline
\end{tabular}

The following update command will execute the body of the \textit{credits\_earned} trigger.
\begin{verbatim}
update takes set grade = 'D' where id = '5' and grade = null
\end{verbatim}
\end{checkboxes}
\end{samepage}

\end{questions}
\end{document}