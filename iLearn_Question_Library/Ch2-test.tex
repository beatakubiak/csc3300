\documentclass[12pt, addpoints, answers]{exam}
\usepackage[left=0.1in, right=0.1in, top=0.6in,bottom=0.1in]{geometry}
\usepackage{enumitem}
\usepackage{amsmath}
\usepackage{tikz}
\usepackage{ulem}
\usetikzlibrary{shapes.geometric, arrows.meta, positioning}

% --- Tikz set for diagram
\tikzset{
    box/.style={rectangle, draw, minimum width=1.8cm, minimum height=1cm, align=left, font=\small},
    doublebox/.style={rectangle, draw, double, double distance=1.5pt, minimum width=2.5cm, minimum height=1cm, align=left, font=\small},
    decision/.style={diamond, draw, minimum width=1cm, minimum height=1cm, align=center, aspect=2, font=\small},
    doubledecision/.style={diamond, double, draw, minimum width=1cm, minimum height=1cm, align=center, aspect=2, font=\small},
    attribute/.style={rectangle, draw, minimum width=1.5cm, minimum height=0.6cm, align=center, font=\small},
    arrow/.style={-Stealth, thick},
    doublearrow/.style={-Stealth, thick, double, double distance=1.5pt},
    dashedline/.style={thick, dashed},
    line/.style={thick},
    doubleline/.style={thick, double, double distance=1.5pt}
}

% --- Make choices use open circles instead of letters ---
\renewcommand{\choicelabel}{\raisebox{0.15ex}{$\bigcirc$}}

% Header/footer formatting
\pagestyle{headandfoot}
\runningheadrule
\firstpageheadrule
\firstpageheader{Database Management Systems}{Test 2}{Fall 2025}
\runningheader{}{Test 2}{}
\runningfooter{}{}{}

\begin{document}
\noindent Name:\ \makebox[3in]{\hrulefill}\hfill Points Scored: \underline{\hspace{3em}} / \numpoints
\begin{questions}

% ------------------ TRUE/FALSE SECTION ------------------
\section*{Part I — True/False}

\begin{minipage}{\linewidth}
\question[2]
True or False: In the University Database, a valid grade for a course section is 'OK'.\par\vspace{0.5em}
\begin{checkboxes}
    \CorrectChoice True
    \choice False
\end{checkboxes}
\end{minipage}\vspace{1em}

\begin{minipage}{\linewidth}
\question[2]
True or False: The University Database allows multiple classrooms in the same building to share the same room number.\par\vspace{0.5em}
\begin{checkboxes}
    \choice True
    \CorrectChoice False
\end{checkboxes}
\end{minipage}\vspace{1em}

\begin{minipage}{\linewidth}
\question[2]
True or False: The University Database allows multiple classrooms in the same building to share the same room number.\par\vspace{0.5em}
\begin{checkboxes}
    \choice True
    \CorrectChoice False
\end{checkboxes}
\end{minipage}\vspace{1em}

\begin{minipage}{\linewidth}
\question[2]
True or False: In the University Database, a \textit{course\_id} can be an empty string ('').\par\vspace{0.5em}
\begin{checkboxes}
    \CorrectChoice True
    \choice False
\end{checkboxes}
\end{minipage}\vspace{1em}

\begin{minipage}{\linewidth}
\question[2]
True or False: A student can be registered for more than one section of the same course in a given semester and year.\par\vspace{0.5em}
\begin{checkboxes}
    \CorrectChoice True
    \choice False
\end{checkboxes}
\end{minipage}\vspace{1em}

\begin{minipage}{\linewidth}
\question[2]
True or False: In the University Database, each instructor can serve as an advisor to at most one student.\par\vspace{0.5em}
\begin{checkboxes}
    \choice True
    \CorrectChoice False
\end{checkboxes}
\end{minipage}\vspace{1em}

\begin{minipage}{\linewidth}
\question[2]
True or False: In the University Database, there can be a classroom in which no course sections were ever taught.\par\vspace{0.5em}
\begin{checkboxes}
    \CorrectChoice True
    \choice False
\end{checkboxes}
\end{minipage}\vspace{1em}

\begin{minipage}{\linewidth}
\question[2]
True or False: The number of tuples in the \textbf{advisor} relation can exceed the number of tuples in the \textbf{studen} relation.\par\vspace{0.5em}
\begin{checkboxes}
    \choice True
    \CorrectChoice False
\end{checkboxes}
\end{minipage}\vspace{1em}

\begin{minipage}{\linewidth}
\question[2]
True or False: The \textbf{section} relation has exactly 3 foreign keys.\par\vspace{0.5em}
\begin{checkboxes}
    \choice True
    \CorrectChoice False
\end{checkboxes}
\end{minipage}\vspace{1em}

\begin{minipage}{\linewidth}
\question[2]
True or False: If \textit{room\_no} attribute of a \textbf{classroom} relation was removed from a primary key \\(not a table though),
then in each building on campus there could be at most one classroom.\par\vspace{0.5em}
\begin{checkboxes}
    \CorrectChoice True
    \choice False
\end{checkboxes}
\end{minipage}\vspace{1em}

\begin{minipage}{\linewidth}
\question[2]
True or False: The University Database maintains a record of all past advisors for each student.\par\vspace{0.5em}
\begin{checkboxes}
    \choice True
    \CorrectChoice False
\end{checkboxes}
\end{minipage}\vspace{1em}

\begin{minipage}{\linewidth}
\question[2]
True or False: In the University Database, courses with the same name can only exist if they are offered by distinct departments.\par\vspace{0.5em}
\begin{checkboxes}
    \choice True
    \CorrectChoice False
\end{checkboxes}
\end{minipage}\vspace{1em}

\begin{minipage}{\linewidth}
\question[2]
It can be the case that in the database there are only the following 2 tuples:\par\vspace{0.5em}

\begin{tabular}{|l|l|l|l|}
\hline
\multicolumn{4}{|c|}{\textbf{student}} \\
\hline
ID & name & dept\_name & tot\_cred \\
\hline
0 & Green & Comp. Sci. & 3 \\
\hline
\end{tabular}
\par\vspace{0.5em}
\begin{tabular}{|l|l|l|l|l|l|}
\hline
\multicolumn{6}{|c|}{\textbf{takes}} \\
\hline
ID & course\_id & sec\_id & semester & year & grade \\
\hline
0 & 3300 & 002 & Spring & 2023 & A \\
\hline
\end{tabular}

\vspace{0.5em}
\begin{checkboxes}
    \choice True
    \CorrectChoice False
\end{checkboxes}
\end{minipage}\vspace{1em}

\begin{minipage}{\linewidth}
\question[2]
True or False: SQL prevents any insert to the database that violates an integrity constraint.\par\vspace{0.5em}
\begin{checkboxes}
    \CorrectChoice True
    \choice False
\end{checkboxes}
\end{minipage}\vspace{1em}

\begin{minipage}{\linewidth}
\question[2]
True or False: A tuple is inserted into the \textbf{advisor} relation only if the value of the \textit{i\_ID} attribute of this tuple is equal 
to the value of the \textit{ID} attribute of some tuple in the \textbf{instructor} relation. \par\vspace{0.5em}
\begin{checkboxes}
    \CorrectChoice True
    \choice False
\end{checkboxes}
\end{minipage}\vspace{1em}

% ------------------ MULTIPLE CHOICE SECTION ------------------

\section*{Part II — Multiple choice (choose one)}

\begin{minipage}{\linewidth}
\question[3]
Assume that there are no tuples in any of the relation instances of the University Database. 
What is the minimal number of tuples that need to be added to the database before one tuple can be added to the table \textbf{section}?\par\vspace{0.5em}
\begin{checkboxes}
    \choice 0
    \CorrectChoice 1
    \choice 2
    \choice 5
\end{checkboxes}
\end{minipage}\vspace{1em}

\begin{minipage}{\linewidth}
\question[3]
With only 5 tuples in the \textbf{instructor} relation, 2 tuples in the \textbf{section} table, and no tuples in the \textbf{teaches} relation, 
what is the maximum number of tuples that can be added to the \textbf{teaches} table without violating any foreign key constraints?\par\vspace{0.5em}
\begin{checkboxes}
    \choice 0
    \CorrectChoice 1
    \choice 2
    \choice 5
\end{checkboxes}
\end{minipage}\vspace{1em}

% ------------------ MULTIPLE SELECT SECTION ------------------
\section*{Part III — Multiple select (select all that apply)}

\begin{minipage}{\linewidth}
\question[4]
Choose superkey(s) of the relation \textbf{takes}. (Select all that apply.) \par\vspace{0.5em}
\begin{checkboxes}
    \choice \{\textit{course\_id, sec\_id, semester, year}\}
    \CorrectChoice \{ID, \textit{course\_id, sec\_id, semester, year}\}
    \CorrectChoice If \{ID, \textit{course\_id, sec\_id, semester, year, grade}\}
    \choice \{ID, \textit{grade}\}
\end{checkboxes}
\end{minipage}\vspace{1em}

\begin{minipage}{\linewidth}
\question[4]
An ISBN (International Standard Book Number) is a unique identifier for books, allowing them to be easily identified and tracked in databases 
and systems worldwide. An ISBN is assigned to each edition and variation (e.g., hardcover, paperback, e-book) of a book.\par\vspace{1em}

Consider the following specification of user requirements for a Library Database: For each book copy, the following data is stored: 
title (\textit{title}), edition (\textit{edition}), number of pages (\textit{page\_no}), barcode (\textit{barcode}), variation (e.g., hardcover, paperback) 
(\textit{variation}), and ISBN (\textit{isbn}). When a member checks out a book copy, its barcode is scanned and linked to the member’s account. 
When the book copy is returned, the barcode is scanned again, and the book is marked as available for borrowing.\par\vspace{1em}

To organize this data, assume the following relational database schema:\par\vspace{1em}

BOOK (\textit{title, edition, pages\_no, variation, isbn})\par
BOOK\_COPY (\textit{variation, isbn, edition, barcode})\par\vspace{1em}

Assume that the attributes variation, isbn, and edition in the BOOK table hold the same data (have the same meaning) as the attributes variation, 
isbn, and edition in the BOOK\_COPY table.\par\vspace{1em}

Which of the following statements are true? (Select all that apply.)
\par\vspace{0.5em}
\begin{checkboxes}
    \CorrectChoice \{\textit{isbn, edition, variation}\} is a candidate key in the BOOK table.
    \CorrectChoice If \{\textit{variation, isbn, edition}\} was the primary key of BOOK, then \{\textit{variation, isbn, edition}\} 
    could be a foreign key in the BOOK\_COPY table.
    \choice \{\textit{ISBN}\} is a candidate key in the BOOK table.
    \CorrectChoice \{\textit{barcode}\} is a candidate key in the BOOK\_COPY table.
\end{checkboxes}
\end{minipage}\vspace{1em}

\begin{minipage}{\linewidth}
\question[4]
In which order can the following commands be invoked on an empty University Database instance? (Select all that apply.)

\vspace{0.5em}

\begin{tabular}{|l|p{0.73\linewidth}|}
\hline
0 & insert into instructor values ('98345', 'Kim', 'Elec. Eng.', '80000') \\
\hline
1 & insert into classroom values ('Taylor', '3128', '70') \\
\hline
2 & insert into department values ('Elec. Eng.', 'Taylor', '85000') \\
\hline
3 & insert into section values ('EE-181', '1', 'Spring', '2009', 'Taylor', '3128', 'C') \\
\hline
4 & insert into student values ('70557', 'Snow', 'Physics', '0') \\
\hline
5 & insert into course values ('EE-181', 'Intro. to Digital Systems', 'Elec. Eng.', '3') \\
\hline
6 & insert into department values ('Physics', 'Watson', '70000') \\
\hline
7 & insert into time\_slot values ('C', 'M', '11', '0', '11', '50') \\
\hline
8 & insert into takes values ('70557', 'EE-181', '1', 'Spring', '2009', 'B') \\
\hline
9 & insert into teaches values ('98345', 'EE-181', '1', 'Spring', '2009') \\
\hline
\end{tabular}
\par\vspace{0.5em}
\begin{checkboxes}
    \CorrectChoice 6, 2, 0, 5, 4, 7, 1, 3, 8, 9
    \CorrectChoice 6, 2, 0, 5, 4, 1, 3, 8, 9, 7
    \choice 6, 2, 0, 5, 4, 7, 3, 1, 8, 9
    \CorrectChoice 6, 2, 0, 5, 4, 7, 1, 3, 9, 8
\end{checkboxes}
\end{minipage}\vspace{1em}

\end{questions}
\end{document}