\documentclass[9pt]{article}

\usepackage[margin=0.2in]{geometry}
\usepackage{enumitem}
\usepackage{courier}
\usepackage{listings}
\usepackage{fancyhdr}

\lstset{
  basicstyle=\ttfamily\footnotesize,
  breaklines=true
}


\begin{document}

\noindent
\textbf{Database Management Systems}%
\hfill%
\textit{Configuration of the development environment for MongoDB}\\
Beata Kubiak, Instructor

\medskip

%-------------- SECTION 1 -------------------%
\section*{1. MongoDB Community Server}

\begin{itemize}[leftmargin=*, itemsep=2pt]
  \item Download the \textbf{ZIP} package (not MSI) from: \texttt{https://www.mongodb.com/try/download/community}.
  \item Extract the ZIP (e.g.\ to \verb|c:\Users\your-username\Desktop|).
  \item Create a data folder, e.g.:
        \begin{flushleft}
        \verb|c:\Users\your-username\Desktop\mongodb-windows-x86_64-8.2.1\data|
        \end{flushleft}
  \item Inside the folder where you extracted the server, there is a subfolder named \texttt{bin}; this is where \verb|mongod.exe| is located.
  \item Start the server from the \texttt{bin} folder:
\end{itemize}

\begin{lstlisting}
mongod.exe --dbpath c:\Users\your-username\Desktop\mongodb-windows-x86_64-8.2.1\data
\end{lstlisting}

\noindent Your server is running now.  
The next time, start the server in the same way: run \verb|mongod.exe| with your data path.

%-------------- SECTION 2 -------------------%
\section*{2. MongoDB Database Tools}

\begin{itemize}[leftmargin=*, itemsep=2pt]
  \item Download the \textbf{ZIP} package (not MSI) from: \texttt{https://www.mongodb.com/try/download/database-tools}.
  \item Extract the ZIP (e.g.\ to \verb|c:\Users\your-username\Desktop|).
\end{itemize}

%-------------- SECTION 3 -------------------%
\section*{3. Sample Databases}

\begin{itemize}[leftmargin=*, itemsep=2pt]
  \item Download the archive:  
        \texttt{https://atlas-education.s3.amazonaws.com/sampledata.archive}
  \item With the server running, restore the archive as follows.
\end{itemize}

\noindent The binary \verb|mongorestore.exe| will be located in a directory such as:
\begin{flushleft}
\verb|c:\Users\your-username\Desktop\mongodb-database-tools-windows-x86_64-100.13.0\bin|
\end{flushleft}

\noindent If \verb|sampledata.archive| is in your Downloads folder, run the command from
\verb|c:\Users\your-username\Downloads| as shown:

\begin{lstlisting}
c:\Users\your-username\Downloads>c:\Users\your-username\Desktop\mongodb-database-tools-windows-x86_64-100.13.0\bin\mongorestore --archive=sampledata.archive
\end{lstlisting}

\noindent This loads all sample databases (including \verb|sample_mflix|).

%-------------- SECTION 4 -------------------%
\section*{4. VS Code Setup}

\begin{itemize}[leftmargin=*, itemsep=2pt]
  \item Install the extension: \textbf{MongoDB for VS Code}.
  \item In the ``MongoDB'' panel, click \textit{Add Connection}.  
        VS Code will show the correct URL: \texttt{mongodb://localhost:27017}.  
        Click \textbf{OK}.
  \item Click \textit{Create MongoDB Playground}.  
        VS Code creates a file with example code; replace it with:
\end{itemize}

\begin{lstlisting}
use('sample_mflix');
db.getCollectionNames();
\end{lstlisting}

\begin{itemize}[leftmargin=*, itemsep=2pt]
  \item Run the Playground file from VS Code; the result should be:
\end{itemize}

\begin{lstlisting}
[
  "movies",
  "theaters",
  "users",
  "embedded_movies",
  "comments",
  "sessions"
]
\end{lstlisting}

\begin{itemize}[leftmargin=*, itemsep=2pt]
  \item After running the file, you may save it (e.g.\ as \verb|mflix_exploration.mongodb.js|) in any folder where you plan to store your Playground scripts. VS Code will treat that folder as the workspace for MongoDB Playgrounds.
\end{itemize}


\end{document}
